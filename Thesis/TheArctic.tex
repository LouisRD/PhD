\section{The Arctic}\label{thearctic}

This chapter introduces the basic notions used by the scientific community to characterise the Arctic Ocean. The covered concepts include (1) a definition of the Arctic, (2) a definition of the Arctic Ocean, circulation of the Arctic Ocean, temperature and salinity profile of the Arctic Ocean and its halocline in section \ref{ocean}, (3) a definition of sea ice, sea ice concentration measured by satellites, definition of sea ice extent, sea ice thickness measurements in section \ref{si}, (4) Arctic sea ice feedbacks in section \ref{feedback}, (5) ocean modelling, sea ice modelling and future projection forcing in section \ref{CCSM}. While the present introduction is succinct, the curious reader is invited to read the references.

The Arctic is widely understood as the northern region of planet Earth. It comes from the Greek word \textit{arktitos} which means \textit{near the Bear} or \textit{northern} referring to the Ursa constellations and Polaris, the North Star. The Arctic has two scientific definitions: (1) everything north of the polar circle and (2) where the average temperature of the warmest month, July, is below $10^\circ C$ \footnote{NSIDC, \textit{https://nsidc.org/cryosphere/arctic-meteorology/arctic.html, last visited December 2017}}. Both Arctic definitions are shown in figure \ref{arctic}. The polar circle marks the northernmost point at which the noon sun is barely visible on the winter solstice - no daylight - and southernmost point at which the midnight sun is barely visible on a summer solstice - no night time. A winter solstice occurs the day with the least sunlight while the summer solstice occurs the day with the most sunlight. The polar circle changes due to planet Earth axial tilt oscillation between $22.1^\circ$ and $24.5^\circ$ over 40 000 years. Right now the polar circle is moving northward at around $15\, m/y$. The second definition follows the tree line, the line separating where trees can still grow and where they cannot grow \citep{treeline}. By that definition, the Arctic is the region where no trees can grow. It is bound to increase with global warming. 

\begin{figure}
\center
\includegraphics[width=0.75\textwidth]{Introduction/arctic.png}
\caption{The two definitions of the Arctic domain: 1) by the polar circle in dashed blues, and 2) by the $10^\circ C$ July average isotherm. This map was contributed by the United States Central Intelligence Agency, published in 2002. The map can be retrieve at the Library of Congress Geography and Map Division, Washington, D.C. or at the web site \textit{https://www.loc.gov/resource/g3270.ct001717/}.}
\label{arctic}
\end{figure}

\subsection{Arctic Ocean}\label{ocean}

The Arctic Ocean is bordered by land from the United States (Alaska), Canada (Canadian Arctic Archipelago or CAA), Finland, Denmark (Greenland), Iceland, Norway (Svalbard), Russia and Sweden, figure \ref{bathymetry}. It is composed of two major basins: the Eurasian basin northeast of Greenland and the Amerasian basin northwest of Greenland divided by the Lomonosov Ridge. The Eurasian Basin is composed of the Amundsen or Fram Basin and the Nansen basin separated by the Gakkel ridge. The Amerasian Basin is composed of the Beauford Basin and the Makarov basin divided by the Alpha ridge. Seven seas share the Arctic Ocean: the Greenland sea that lies east of Greenland, the Beaufort Sea between Canada and Alaska and fives seas that border Russia. From east to west they are the Barents Sea, Kara Sea, Laptev Sea, East Siberian Sea and the Chukchi Sea which also borders Alaska. The Arctic Ocean has the highest ratio by far of continental shelf area over ocean area with a ratio of one third, making it unique.  The mean depth of the Arctic Ocean is about one kilometre with a maximum depth of $5.5 \, km$. 

\begin{figure}
\center
\includegraphics[width=0.75\textwidth]{Introduction/bathymetryarctic.jpg}
\caption{International Bathymetric Chart of the Arctic Ocean annotated with the names of basins, ridges and shelves \citep{bathymetryarcticocean}.}
\label{bathymetry}
\end{figure}

The Arctic Ocean circulation is depicted in figure \ref{circulation}. Warm Atlantic waters enter the Arctic Ocean by the West Spitzbergen current west of Svalbard and by the North Cape current in the Barents Sea. The Atlantic waters are heavier than the Arctic Ocean surface waters being saltier though warmer. They sink and turn counterclockwise at the edge of the Arctic Ocean. The Pacific waters enter the Chukchi Sea and follow the Alaskan and Canadian coast into the Canadian Arctic Archipelago. The interior of the Amerasian Basin is dominated by the Beaufort Gyre turning clockwise. The transpolar drift takes waters through the Arctic Ocean following the Lomonosov Ridge through the Fram Basin exiting the Arctic Ocean as the East Greenland current. 

\begin{figure}
\center
\includegraphics[width=0.75\textwidth]{Introduction/circulationao.png}
\caption{Circulation map of the Arctic Ocean. Red arrows show warm currents and blue arrows cold currents. The green region is for the Sub-Arctic, the purple region is for the Low Arctic and the blue region is for the high arctic \citep{circulation}.  }
\label{circulation}
\end{figure}

The vertical temperature-salinity structure of the Arctic Ocean exhibits the presence of a cold halocline layer, figure \ref{tsprofile}. The observed temperature-salinity profile is constant at $-1.8^\circ C$ and $33.5 \, PSU$ over the first 50 meters. This continuity arises from mechanical surface forcings such as wind or ice motion causing turmoil and mixing over the first 50 meters. It is called the surface mixed layer.  Between 50 and 200 metres, the ocean temperature stays at freezing point but the salinity increases. This layer is the halocline. The increased salinity makes the water denser, stabilizing the water column which diminishes the vertical mixing insulating the sea ice from the warm Atlantic waters between 200 and 500 metres at $1 ^\circ C$ and $35\, PSU$ \citep{steele1998}. \citet{halocline} showed that the halocline is created by the salt rejection during ice formation. This salt rejection increases the salinity of cold already saline surface waters causing its density to increase and sink between 50 and 200 metres.

\begin{figure}
\center
\includegraphics[width=0.75\textwidth]{Introduction/TSprofile.png}
\caption{Schematic representation of the temperature and salinity structure in the upper Arctic Ocean and its maintenance. Taken from \citet{halocline}.}
\label{tsprofile}
\end{figure}

%\citet{halocline} showed that the halocline could not be maintained by cooled Atlantic waters which would lead to saltier water than observed in the cold halocline. The salt rejection from ice formation could explain the cold halocline. The ice formation over the Barents sea shelf and the Kara Sea can explain the presence of a halocline in their vicinity and also for the Eurasian basin. They lacked data for any conclusion over the Laptev sea. The East Siberian sea does not produce enough sea ice to feed the halocline of the Canadian basin. Their theory is that the East Siberian sea shelf might produce enough sea ice to feed the Canadian basin. The Chukchi Sea does not form enough sea ice but the presence of surface divergence keeping the ocean's surface ice free allowed for more formation of thin ice. This thin ice rejects salt more efficiently than growing thick ice. It can then explain the presence of a cold halocline over the Chukchi Sea. 


\subsection{Sea ice}\label{si}

Ice formed from sea water contains salt therefore sea ice is a complex material. In this section, a brief introduction of sea ice as a material is presented. Section \ref{sie} defines sea ice extent and presents measurements of it. Section \ref{sit} presents sea ice thickness measurements. 

Sea ice is a mixture of freshwater solid ice crystals and interstitial liquid salty brine \citep{porousice}. Brine trapped in sea ice can escape and reach the ocean by five processes: initial salt rejection, salt diffusion, brine expulsion, gravity drainage and flushing. \citet{saltNotz} explored all five options using analytical solutions from a mushy layer \citep{mushy}, numerical solutions from a one-dimensional enthalpy model \citep{1dNotz} and observations \citep{obssalt,thesisNotz}. They found that only gravity drainage during winter and flushing during summer contributed to salt rejection. During winter, as the surface sea ice temperature decreases, the interstitial liquid brine becomes denser creating a brine-density unstable profile. This heavier brine will push down resulting in convection replacing salty sea ice brine by ocean water. The flushing happens in summer when the ice becomes permeable. \citet{rule5} found that the ice becomes permeable at a brine volume fraction of $5\%$, temperature of $-5 ^\circ C$ and salinity of 5 parts per thousand. This is known as the law of fives. When the sea ice becomes permeable, all water inside and over the ice is flushed down into the ocean leaving the ice brine free. 

% SEA ICE

The Arctic Ocean is covered by sea ice. Sea ice concentration (SIC) is measured by satellite equipped with a microwave radiometer \citep{satellite}. The radiometer records the intensity (or brightness temperature $T_B$) of electromagnetic radiation for a specified frequency~$\nu$,
\begin{equation}
T_B = \epsilon \, T_S e^{-\tau} + \int_0^\tau T(z)\zeta(z) e^{-\tau +\tau'(z)} d\tau' + (1-\epsilon) \kappa e^{-\tau} \int_0^\tau T(z)\zeta(z)e^{-\tau'(z)} d\tau'(z),
\end{equation}
$\epsilon$ is the emissivity of the surface, $T_S$ is the surface temperature, $\tau$ is the atmospheric opacity from the surface to the satellite, $T(z)$ is the temperature at height $z$, $\zeta(z)$ is the emittance at height $z$, $\tau'(z)$ is the atmospheric opacity from the surface to height $z$, $\kappa$ is an estimate of the diffusiveness of the surface reflection. The first term of the right-hand side of the equation is the surface emission for frequency $\nu$ reaching the satellite which is often the dominant source. The second term is the emission from the atmospheric column reaching the satellite. The third term represents the downwelling atmospheric column radiation reflected at the surface and travelling back up to the satellite. For a studied region, the brightness temperature can be rewritten as the linear sum of ice emission  and ocean emission,
\begin{equation}
T_B = T_O C_O + T_I C_I,
\end{equation}
where $T_O$ is the brightness temperature of the ocean, $C_O$ is the fraction of the region covered by ocean, $T_I$ is the brightness temperature of the ice, $C_I$ is the fraction of the region covered by sea ice or sea ice concentration. Since $C_O + C_I = 1$, we can rewrite the last equation as
\begin{equation}
C_I = \frac{T_B - T_O}{T_I - T_O}.
\end{equation}
Measuring $T_B$, $T_O$ and $T_I$ is challenging since all the terms depends on surface temperature, atmospheric opacities and emissivity. To do so, one usually analyze three different frequencies. During winter or when the weather is dry, changes in emissivity and surface temperature can be taken into account effectively leaving only a $5\%$ error on the sea ice concentration under optimal conditions. When the ice is scattered, the many types of surfaces - ocean, snow, first year ice, older ice -  over the studied region increases the error up to $15\%$. During summer time, the emissivity of snow and ice is unpredictable and melt ponds are seen as ocean resulting in an error that can reach $25\%$. The error can be higher than $25\%$ during severe storms and harsh weather. At the time of writing this thesis, the interior of the Arctic Ocean is still covered by highly concentrated sea ice allowing a satisfying level of confidence in satellite measurements. The National Snow and Ice Data Center (NSIDC) provides the satellite observations for the sea ice concentration. 

\subsubsection{Sea ice extent}\label{sie}

To quantify how sea ice evolves, Arctic scientists study sea ice extent (SIE) over sea ice area (SIA). Sea ice extent is defined as the sum of all the studied region or grid cell area with more than $15\%$ sea ice concentration while the sea ice area is the product of sea ice concentration and area. The value of $15\%$ has been chosen based on the seven measures of ice-edge features by aircraft carrier reported in \cite{iceedge}. They measured $11\%$, $20\%$, $20\%$, $13\%$, $15\%$, $15\%$, $15\%$ sea ice concentration which averages at $15\%$. Studying sea ice extent is useful during summer where the surface melt and melt ponds appear as water even if sea ice is still present under a thin liquid layer. Because sea ice extent does not discard those regions unlike sea ice area, it gives a more reliable measurement of the total coverage of Arctic sea ice. Sea ice extent always covers a larger area than sea ice area, figure \ref{nsidc}. Their evolution is close but not precisely the same. The four black dots display the four sea ice extent minimums of the $21^{st}$ century: 2002 \citep{2002}, 2005, 2007\citep{EOST:EOST16266} and 2012 \citep{2012}.

\begin{figure}
\center
\includegraphics[width=0.75\textwidth]{Introduction/nsidc.jpg}
\caption{Sea ice extent in blue and sea ice area in red. Both curves are bound by their maximum or March area and minimum or September area. The four black dots display the four sea ice extent minima, 2002, 2005, 2007 and 2012. Data provided by the NSIDC \cite{data}.}
\label{nsidc}
\end{figure}


% SEA ICE THICKNESS.
\subsubsection{Sea ice thickness}\label{sit}

Sea ice thickness can be measured from different methods: (1) Upward Looking Sonar (ULS), (2) ElectroMagnetic (EM) sounding, (3) satellites or (4) direct measurement. The ULS can be attached to a submarine or moored on the sea bed. It calculates its own depth from the ambiant pressure and the distance to ice from sonar reflection on the ice base \citep{ULS}. The EM sounding device - called an EM bird - is towed by a helicopter and measures its distance to the sea ice by a laser altimeter and its distance to the ocean using a frequency unaffected by sea ice \citep{EMbird}. The resulting ice thickness has an error of $\pm 0.1 \,m$. The main satellites investigating sea ice are CRYOsat and ICESat. CRYOsat is developed by the European Space Agency (ESA) and ICESat by NASA (United States of America). They both calculate the distance from the first solid or liquid surface to the satellite, respectively \cite{cryosat} and \cite{icesat}. By using sea surface height measurement from the satellite, it is possible to transform the sea ice height in sea ice thickness. \cite{subplusat} combined the results from ULS, EM sounding and satellites. They found that most measurements were in acceptable agreement with the annual mean sea ice thickness between 2000 and 2012. The annual mean sea ice thickness between 2000 and 2012 is roughly about 5 metres at the Northern part of Canada and decreases toward Russia down to one metre. They also found a rapid decrease in ice thickness. From 2000 to 2012, the yearly averaged ice thickness has declined by $0.58 \pm 0.07 \, m/decade$ which represents a loss of $34\%$ while the September thickness declined by $50\%$. Using data from 1975 to 2012, the yearly averaged thickness decreased by $65\%$ and the September thickness by $85\%$.

The worrying decay of ice in the Arctic inspired Andy Lee Robinson from Climate State to plot the dramatic Arctic Death Spiral (figure \ref{ads}). It shows the decrease in monthly mean Arctic sea ice volume for each month from 1979 to 2017. Every month, approximately $15\cdot 10^3 km^3$ of sea ice volume is lost over the time span. The Pan-Arctic Ice Ocean Modelling and Assimilation System (PIOMAS) provided the data. It consists of a coupled ocean-ice climate model. The ocean part is the Parallel Ocean Program (POP) model developed at Los Alamos \citep{SmithPop2010}. The ice model is a thickness-enthalpy distribution (TED) sea ice model using viscous plastic rheology \citep{PIOMAS}. The atmospheric forcing is derived from NCEP-NCAR reanalysis \citep{NCEPNCAR}.  PIOMAS showed reasonable agreement with observations.
\begin{figure}[H]
\center
\includegraphics[width=0.75\textwidth]{Introduction/ADS.jpg}
\caption{Arctic death spiral showing the evolution for each monthly sea ice volume between 1979 and 2017.}
\label{ads}
\end{figure}

% FEEDBACK

\subsection{Feedback}\label{feedback}

The Arctic reacts more than the rest of the world to global warming because of overwhelming positive feedback processes \citep{ACIA,HollandBitz2003}. A positive feedback happens when a change in a system amplify itself. For example, sea ice loss brings more sea ice loss. A negative feedback happens when a change in a system erodes itself. Four of the most important feedback are presented in this section: the ice-albedo feedback, thinner ice feedback, cloud-ice feedback and the thermohaline feedback. 

The ice-albedo feedback is the most important feedback. Sea ice is highly reflective to shortwave solar radiation whereas ocean is a good absorber of solar radiation. The albedo or reflectivity of ocean water is 0.1, between $0.3 - 0.4$ for land ice, between $0.5 - 0.7$ for sea ice and between $0.75 - 0.95$ for snow \citep{Ahrens:2009fk,Sknow:2013fk}.  If the area of open water increases,  the total absorption of solar radiation by the Arctic Ocean will be higher. This energy will lead to an increase in ocean surface temperature. A warmer ocean melts more sea ice, and therefore creates more open water. 

A thinner sea-ice cover is weaker, deforms more easily and drifts faster for a given wind forcing. Sea-ice deformations primarily occur along leads where large gradient in the sea-ice velocity (shear) and some divergence or convergence are present. Divergence in the ice pack leads to more open water, more evaporation, and consequently more storms. Therefore, a stormier Arctic leads to more sea-ice deformation and faster drift. \citet{Rigor:2002fk} showed that when the large-scale atmospheric circulation is more cyclonic with storms penetrating farther into the eastern Arctic, the wind pattern are such that it advects readily thick multi-year ice from the central Arctic through the Fram Strait. This was the case in the eighties and early nineties when a dramatic loss of multi-year ice was recorded \citep{Johannessen1999}. 

The next mechanism is the cloud-ice feedback. If there are more clouds, there is less solar energy income during the summer. This leads to less summer melt and generally to a thicker sea-ice cover. However, liquid water clouds have a strong greenhouse effect and also lead to an increase in the downwelling surface longwave radiation and this, during the whole year \citep{Shupe:2004fk,Gorodetskaya:2008kx}. The resulting cloud-ice reaction is overall a positive feedback.

The last feedback is linked with the thermal insulation of the ocean. When the ice is thin, it is easier to form ice hence a negative feedback. New formed ice rejects brine. This extra brine builds a stronger cold halocline layer which insulates the top part of the Arctic Ocean and its sea ice from the warm Atlantic waters just as discussed earlier. 

When global climate models include all the feedback processes, they predict the largest change of surface air temperature at high latitudes \citep{HollandBitz2003}. The ice-albedo feedback is likely to account for much of the warming. The Arctic is a magnifying glass for global warming. This is called polar amplification.

% IMPACT OF SEA ICE LOSS 
With this amplification, the Arctic Ocean could become seasonally ice free by 2050 \citep{Holland2006fa, GRL:GRL23061, GRL:GRL29477, ISI:000309414600002}. All this occurs to the detriment of any ice-dependent species such as narwhal, seal, polar bear, etc. \citep{Laidre07122008, Stirling2012pb,Tremblay:212fk, Norris2002} A seasonal ice-free Arctic will lead to local warming of adjacent land and likely to an acceleration of permafrost melting which will release more greenhouse gas \citep{permafrost,GRL:GRL24620}. Since the density of melted ice is not the same as the ocean, a fully melted Arctic would increase the sea level by and extra $47 \, mm$ \citep{SLRice} compared to $7 \, m$ for a fully melted Greenland and $60 \, m$ for a fully melted Antarctica estimated by \cite{GAslr}. Vulnerabilities will strengthen and develop in Inuit communities \citep{ford2006,ford2009}. On a more positive side, the maritime shipping routes through the Arctic Ocean will open reducing transport time, cost and pollution \citep{2016melia}.
% Explication de sea level rise?

% CCSM
\subsection{Community Climate System Model}\label{CCSM}

In order to understand and anticipate the advercities of climate change over the Arctic, the scientific community uses Global Climate Models (GCMs). GCMs modelise the equations of climate on a global scale. Simulations start in the past up to present time and then uses predictions for the green house gases future evolution in order to estimate future climate. This thesis will focus on the results of the Community Climate System Model (CCSM) versions 3 and 4. The CCSM is a coupled climate model consisting of four modules, one for atmosphere, one for ocean, one for land, one for sea ice, and a coupler assuring communication between each module. The general results of the CCSM3 are described in \cite{Collins2006ccsm3} and in \citet{gent2011} for the CCSM4. The CCSM3 uses the Community Atmospheric Model version 3 (CAM3) \citep{cam3a,cam3b},  the Parallel Ocean Program (POP1) \citep{Smith:2004uq},  the Community Sea-Ice Model version 5 (CSIM5) \citep{Briegleb:2004kx} and the Community Land Model version 3 (CLM3) \citep{clm3a,clm3b,Bonan:2002uq}. The CCSM4 uses the Community Atmospheric Model version 4 (CAM4) \citep{Neale:2010qf}, the Parallel Ocean Program version 2 (POP2) \citep{SmithPop2010, Danabasoglu:2012zr}, the Los Alamos sea ice model version 4 (CICE4) \citep{Hunke:2008ly} and the Community Land Model version 4 (CLM4) \citep{clm4a,clm4b,Lawrence:2011ys}.

We will narrow our attention to the ocean and sea ice components of the models in sections \ref{oceanCCSM} and \ref{siCCSM} respectively. For each component, the basic equations used in both versions 3 and 4 as well as their differences are presented. A succinct presentation of the future scenarios used by the CCSM to forecast the climate up to 2100 follows in section \ref{fs}. The following model discussion is technical and might be difficult to follow for beginners. It is more informative than explicative. 

\subsubsection{CCSM ocean component - POP model}\label{oceanCCSM}

A derivation of the fluid equations and approximations can be found in \cite{vallis}. The POP model solves the Navier-Stokes equations for $u$, the eastward velocity, and $v$, the northward velocity, on a sphere for a thin stratified fluid under the hydrostatic and Boussinesq ($\rho=\rho_0$) approximation \citep{SmithPop2010}:
\begingroup \belowdisplayskip=0pt \abovedisplayskip=0pt
\begin{gather}
\frac{\partial}{\partial t} u + \mathcal{L} (u) - (u v \tan \phi)/a - f v = -\frac{1}{\rho_0 a \cos \phi} \frac{\partial P}{\partial \lambda} + \mathcal{F}_{Hx}(u,v)+\mathcal{F}_V(u)\\ \label{equ}
\frac{\partial}{\partial t} v + \mathcal{L} (v) - (u^2 \tan \phi)/a + f u = -\frac{1}{\rho_0 a} \frac{\partial P}{\partial \phi} + \mathcal{F}_{Hy}(u,v)+\mathcal{F}_V(v)\\ \label{eqv}
%
\mathcal{L}(\alpha) = \frac{1}{a \cos\phi} \big[\frac{\partial}{\partial \lambda}(u\alpha)+\frac{\partial}{\partial \phi}(cos\phi v \alpha) \big] + \frac{\partial}{\partial z}(w\alpha)\\
%
\mathcal{F}_{Hx}(u,v) = A_M \big\{ \nabla^2 u + u(1-\tan^2\phi)/a^2 - \frac{2 \sin\phi}{a^2 \cos^2\phi}\frac{\partial v}{\partial \lambda}\big\}\label{W1}\\
%
\mathcal{F}_{Hy}(u,v) = A_M \big\{ \nabla^2 v + v(1-\tan^2\phi)/a^2 + \frac{2 \sin\phi}{a^2 \cos^2\phi}\frac{\partial u}{\partial \lambda}\big\}\label{W2}\\
%
\nabla^2\alpha = \frac{1}{a^2 \cos^2\phi}\frac{\partial^2 \alpha}{\partial \lambda^2}+\frac{1}{a^2 \cos^2\phi}\frac{\partial}{\partial \phi}\big(\cos\phi \frac{\partial\alpha}{\partial \phi}\big)\\
%
\mathcal{F}_V(\alpha) = \frac{\partial}{\partial z}\mu\frac{\partial}{\partial z}\alpha,
\end{gather}
\endgroup
where $\lambda$ is the longitude, $\phi$ is the latitude, $z=r-a$ is the depth relative to mean sea level with $a=6.37122\cdot10^6\, m$ being the Earth's radius, $g=9.80616\, m^2/s$ is the acceleration due to gravity, $f=2 \cdot \sin \phi$ is the Coriolis parameter, $\rho_0 = 1.026 g/cm^3$ is the background density of seawater, $w$ is the vertical or radial velocity, $P$ is the pressure, $\Theta$ is the potential temperature, $S$ is the salinity, $A_M$ is the horizontal diffusion coefficient and $\mu$ is the vertical mixing coefficient. The first term on the left-hand side of equations \eqref{equ} and \eqref{eqv}, $\frac{\partial}{\partial t} u$ and $\frac{\partial}{\partial t} v$,  is the temporal evolution of the horizontal velocities. The second and third terms on the left-hand side represent the advection of the velocities. The fourth and last term on the left-hand side represents the Coriolis force. The first term of the right-hand side of \eqref{equ} and \eqref{eqv} takes into account the pressure driven circulation. The second and third terms of the right-hand side contains velocity diffusion, both horizontal and vertical respectively. The second and third terms in brackets in equations \eqref{W1} and \eqref{W2} ensure that no stresses are generated due to solid-body rotation in a shallow layer \citep{1972Williams}. Wind forcing is applied as a boundary condition of the variable $F_V$. Bottom and lateral boundary conditions applied in POP are no-flux for tracers (zero tracer gradient normal to boundaries) and no-slip for velocities ($u=v=w=0$). The main parameterizations added are a mesoscale (sub-grid) eddy parameterization \citep{Gent:1990zr} and a vertical mixing K-profile parameterization (KPP) \citep{ROG:ROG1432}

The Boussinesq approximation stipulates that density changes has no effect on the circulation. Therefore, the continuity equation can be used to calculate the vertical velocity as a residual only after the horizontal velocities has been solved,
\begin{equation}
\mathcal{L}(1) = \frac{1}{a \cos\phi} \big[\frac{\partial}{\partial \lambda}(u)+\frac{\partial}{\partial \phi}(\cos\phi v) \big] + \frac{\partial}{\partial z}(w) = \nabla \cdot \vec{u} = 0.
\end{equation}

The hydrostatic approximation states that the pressure gradient is balanced by buoyancy only,
\begin{equation}
\frac{\partial p}{\partial z} = -\rho g.
\end{equation}

The scalar variables or tracers such as potential temperature and salinity are calculated using the tracer transport equation,
\begingroup \belowdisplayskip=0pt \abovedisplayskip=0pt
\begin{gather}
\frac{\partial}{\partial t}\varphi + \mathcal{L}(\varphi) = D_H(\varphi)+D_V(\varphi)\\
D_H(\varphi) = A_H\nabla^2\varphi\\
D_V(\varphi) = \frac{\partial}{\partial z}\kappa\frac{\partial}{\partial z}\varphi,
\end{gather}
\endgroup
where $A_H$ is the horizontal diffusion coefficients and $\kappa$ is the vertical mixing coefficients. The surface heat fluxes or fresh water fluxes are applied as surface boundary conditions to vertical diffusive terms $D_V$. The first term of the left-hand side of the tracer equation is the temporal evolution of the tracer. The second term represents the tracer advection. The left-hand side contains the horizontal and vertical tracer diffusion.

Once the potential temperature and the salinity are solved, the CCSM calculates the density as a state function depending on potential temperature, salinity and depth. The resulting density field is used to calculate instabilities which can induce motion. 

At the North Pole, $\phi = 90^\circ$, every term with a $\frac{1}{cos \phi}$ diverges. At the North Pole, latitudes and longitudes do not have defined derivatives. To avoid any problem with divergent terms, the CCSM chooses a grid with a displaced pole over Greenland (figure \ref{polegrid}) with a nominal $1^\circ$ resolution. It is also possible to choose a displaced pole over Canada or a tri-pole grid. To modify the POP model basic equations on a sphere to a different grid, we must use results from differential geometry \citep{EDG}. Without going into details, the goal is to start with the equations using coordinates $(\lambda,\phi,z)$ and end with the equations written with the new coordinates $(q_x,q_y,z)$. The differential length element, $ds$, is given by:
\begin{align}
ds^2 = \xi_\lambda^2 + \xi_\phi^2 + dz^2 &= \sum_{i,j=1}^2 h_{i\,j}^2 dq_i dq_j + dz^2\\
h_{i\,j} &= \sum_{k=1}^2 \frac{\partial \xi_k}{\partial q_i}\frac{\partial \xi_k}{\partial q_j},
\end{align}
where $\xi_k$ is the infinitesimal distance given by an infinitesimal change in $\lambda$ ($k=1$) or $\phi$ ($k=2$), $h_{i\,j}$ is the metric coefficient of the new grid which depends on the local curvature of the new set of coordinates. This metric holds all the required information to rewrite the terms and operators of the equations. 

 \begin{figure}[H]
\center
\includegraphics[width=0.75\textwidth]{Introduction/greenland_pole_grid.jpg}
\caption{CCSM grid with the North Pole displaced over Greenland.}
\label{polegrid}
\end{figure}

The POP model uses a staggered Arakawa B-grid with tracers (scalar variables) at the centre of a cell and velocities $u_x$ and $u_y$ at the corners of the cell mid height as one can see on the left picture of figure \ref{Bgrid}. From this grid, two sub grid are considered: the T-grid (tracer-grid) and the U-grid (velocity grid). The T-grid has the tracers at the centre of the cell with velocities at its corners. The U-grid has velocities $u_x$ and $u_y$ at the centre of the grid and tracers at its corners. The vertical velocity,$w$, can be found at two locations. The first location is at the highest north-east corner of the cell and the second location is mid height on the north-east corner where the horizontal velocities are as one can see on the central panel of figure \ref{Bgrid}. The first location facilitates the advection calculation of the tracers while the second location facilitates the advection calculation of the velocities hence its superscript, $w^u$. The vertical discretization increases with depth as illustrated in the right panel of figure \ref{Bgrid}.

\begin{figure}[H]
\center
\includegraphics[width=0.32\textwidth]{Introduction/grid1.png}
\includegraphics[width=0.32\textwidth]{Introduction/grid2.png}
\includegraphics[width=0.32\textwidth]{Introduction/grid3.png}
\caption{Left) Horizontal discretization of the grid showing the locations of tracers, velocities and lengths. Center) Vertical locations of tracers and velocities. Right) Vertical discretization. Taken from \citet{SmithPop2010}.}
\label{Bgrid}
\end{figure}

The main improvements between the POP and the POP2 models are the increase of 40 vertical levels to 60 with a new bottom topography, the inclusion of the opening of Nares Strait, a near-surface eddy flux parameterization \citep{2008ferrari}, vertically varying diffusive coefficients \citep{2006ferreira}, a submesoscale parameterization \citep{2008fka,2008fkb,2011fk}, a deep overflow parameterization \citep{briegleb2010}, and an abyssal tidally driven mixing parameterization \citep{2009jayne}. \citet{2012jahn} showed that the temperature-salinity profile of the CCSM4 worsens compared to CCSM3 compared to observations. The CCSM3 temperature-salinity profile agrees satisfactorily with observations. The CCSM4 has has issues at all depth. The surface waters are too cold and too saline. Between 50 and 300 metres, the salinity is too low compared to observations while the temperature profile agrees with the observations. From 400 metres until the ocean floor, the temperature is too high and stays $1^\circ C$ too high until the ocean floor with a salinity profile slightly too high. The observed warm Atlantic waters are located at 400 metres while the simulations place them at 750 meters. The simulated Atlantic waters in the Arctic Ocean are too deep and too warm. The CCSM3 temperature-salinity profile agreed better with observations. The general shape of the ocean circulation agrees with observations specially the inflow from the Fram Strait and the Barents and Kara seas, the cyclonic boundary current around the marginal seas, the Canadian Basin gyre, the return flow along the Lomonosov Ridge and the return current west and north of Svalbard. The velocities are too low at the gates of the Arctic Ocean but the temperature agrees with observations leading to wrong heat fluxes entering and exiting the Arctic Ocean. 

% Sea ice model
\subsubsection{CCSM sea ice component - CICE model and CSIM}\label{siCCSM}


From \citet{Hunke:2008ly}, the fundamental equation solved by CICE is \citep{Thorndike:1975fk}
\begin{equation}\label{ciceeq}
\frac{\partial g}{\partial t} = -\nabla \cdot (g\vec{u})-\frac{\partial}{\partial t}(fg)+\psi,
\end{equation}
where $g$ is the ice thickness distribution function defined as $g(\vec{x},h,t) dh$ being the fractional area covered by ice in the thickness range $(h; h + dh)$ at a given time and location, $\vec{u}$ is the horizontal ice velocity vector, $f$ is the rate of thermodynamic ice growth and $\psi$ is a ridging redistribution function. 

This equation is discretized horizontally on the same horizontal grid as the ocean model and in discrete thickness categories. The thickness does not become discrete, the ice still changes continuously inside its thickness category. As an example, consider a two thickness category model where the first category ranges from $0$ to $1\, m$ and the second category includes any ice thicker than $1\, m$. This model allows a cell with a section with sea ice less thick than $1\, m$ and another section with ice thicker than $1\, m$. At the beginning of winter, the first category could start with $30\%$ of $50\,cm$ and $70\%$ of $2\,m$ sea ice. Through the cold season, both category will grow, possibly at different rate. When the first category grows up to $1\,m$, its sea ice will be transferred to the next category. The thickness discretization allows a number of different thicknesses inside a same cell. Without it, a cell would have one averaged thickness. 

The CICE models follows three steps at each time step to solve the fundamental equation \eqref{ciceeq}: 1) calculation of the sea ice velocities, $u$ and $v$, 2) transport and ridging of sea ice accompanied with tracer transport, 3) thermodynamic processes growing and melting sea ice, affecting the temperature and salinity profiles. Each steps is explained more thoroughly in the next paragraphs. 

The CICE model starts by solving ice dynamics using an elastic-viscous-plastic (EVP) model \citep{Hunke:1997vn}. The EVP models treats the ice pack as a plastic material under typical stress conditions but behaves as an elastic-viscous fluid where strain rates are small. The equations of the sea ice dynamics are
\begingroup \belowdisplayskip=0pt \abovedisplayskip=0pt
\begin{gather}
m\frac{\partial \vec{u}}{\partial t} = \nabla\cdot\sigma + \vec{\tau_a} + \vec{\tau_w} - m\,f\, \hat{k}\times \vec{u} - m\,g\,\nabla H_0\\
\vec{\tau_a} = c_a \cdot \rho_a \cdot |\vec{U}_a| \cdot (\vec{U}_a cos\phi + \vec{k}\times \vec{U}_a sin\phi),\\
\vec{\tau_w} = c_w \cdot \rho_w |\vec{U}_w - \vec{u}|\big[ (\vec{U}_w - \vec{u}) \cos\lambda + \vec{k}\times(\vec{U}_w - \vec{u})sin\lambda\big],
\end{gather}
\endgroup
where m is the combined mass of ice and snow per unit area, $\vec{\tau_a}$ and $\vec{\tau_w}$ are the wind and ocean stresses, $\sigma$ is the internal stress tensor containing the EVP characteristics of the ice model, $m\,f\, \hat{k}\times \vec{u}$ is the Coriolis force and $m\,g\,\nabla H_0$ is the effect of the sea surface slope. The details of the internal stress tensor can be found in \citet{Hunke:2008ly}.

With the updated velocities, CICE calculates the transport and ridging of sea ice through three variables: 1) sea ice concentration $a_{i\,n}$ for a given grid cell and for a given thickness category $n$, 2) sea ice volume $v_{i\,n} = a_{i\,n}\cdot h_{i\,n}$ where $h_{i\,n}$ is the ice thickness and 3) internal ice energy for a given vertical layer $k$, $e_{i\,n\,k}=\frac{v_{i\,n}}{N_i}\cdot q_{i\,n\,k}$ where $N_i$ is the total number of ice categories and $q_{i\,n\,k}$ is the ice layer enthalpy which is minus the amount of energy required to melt a unit volume of ice and raise its temperature to $0^\circ C$. The transport equations are:
\begingroup \belowdisplayskip=0pt \abovedisplayskip=0pt
\begin{gather}
\frac{\partial a_{i\,n}}{\partial t}+\nabla\cdot (a_{i\,n} \vec{u_i}) = 0,\\
\frac{\partial v_{i\,n}}{\partial t}+\nabla\cdot (v_{i\,n} \vec{u_i}) = 0,\\
\frac{\partial e_{i\,n\,k}}{\partial t}+\nabla\cdot (e_{i\,n\,k} \vec{u_i}) = 0.
\end{gather}
\endgroup
The equations of ridging, variable $\psi$ in equation \eqref{ciceeq}, are described in \citet{JGRC:JGRC10227}. The CICE model uses elastic-viscous-plastic (EVP) equation for sea ice. A similar set of equations is prescribed in the CICE model for snow over sea ice.

Tracers transport is calculated using the updated velocities,
\begingroup \belowdisplayskip=0pt \abovedisplayskip=0pt
\begin{gather}
\frac{\partial a_{i\,n}T_n}{\partial t}+\nabla\cdot (a_{i\,n} T_n\vec{u_i}) = 0,\\
\frac{\partial v_{i\,n}T_n}{\partial t}+\nabla\cdot (v_{i\,n}T_n \vec{u_i}) = 0,\\
\frac{\partial v_{s\,n}T_n}{\partial t}+\nabla\cdot (v_{s\,n}T_n \vec{u_i}) = 0,
\end{gather}
\endgroup
Where $T_n$ denotes the tracer value for the sea ice thickness category $n$.

Once the transport is calculated, the CICE model considers thermodynamics of sea ice. The thermodynamics of sea ice include surface and bottom forcing, temperature and salinity changes, melt and growth. The surface forcings are given by
\begin{equation}
F_0 = F_s +F_l+F_{lw}+(1-\alpha)(1-i_0)F_{sw},
\end{equation}
where $F_s = C_s(\Theta_a-T_{sf})$ is the sensible heat flux with $C_s$ being a nonlinear turbulent heat transfer coefficient and $\Theta_a$ being the atmosphere's surface potential temperature, $F_l = C_l (Q_a-\frac{q_1}{\rho_a}\exp(\frac{-q_2}{T_{sf}}))$ is the latent heat flux with $C_l $ being a nonlinear turbulent heat transfer coefficient and $Q_a$ being the atmosphere's specific humidity at the surface, $F_{lw} = \epsilon F_{lw\,a} - \epsilon\sigma_{SB}T_{sf}^4$ is the longwave heat flux with $\epsilon=0.95$ being the emissivity of snow or ice, $F_{lw\,a}$ being the longwave atmospheric radiation reaching sea ice and $\sigma_{SB}$ is the Stefan-Boltzmann constant, $\alpha$ is the albedo of sea ice, $1-i_0$ is the fraction of absorbed shortwave flux that penetrates into the ice, $F_{sw}$ is the incoming short wave radiation. The albedo and penetrative fraction of short wave radiation depend on sea ice thickness, surface ice optical properties, presence of melt pond or snow. They are calculated by the model. The bottom heat flux between the ocean and sea ice, $F_{bot}$, is given by
\begin{equation}
F_{bot} = - \rho_w \cdot c_w \cdot c_h \cdot u_* \cdot (T_w - T_{fr}),
\end{equation}
where $\rho_w$ is the water density, $c_w$ is the water heat capacity, $c_h$ is the heat exchange coefficient, $u_* = \sqrt{|\tau_w|/\rho_w}$ is the friction velocity, $|\tau_w|$ is the norm of the shear stress given by $\rho_w \sqrt{\frac{\partial u}{\partial y}^2+\frac{\partial v}{\partial x}^2}$, $T_w$ is the surface water temperature, $T_{fr}=-1.8^\circ C$ is the freezing temperature of ocean water.

The vertical salinity profile is prescribed. The midpoint salinity $S_{i\,k}$ in each ice layer $k$ is given by
\begin{equation}
S_{i\,k} = \frac{1}{2}S_{max}[1-\cos(\pi z^{\frac{a}{z+b}})],
\end{equation}
where $z = (k-1/2)/N_i$, $S_{max}$= 3.2 psu and $a=0.407$ and $b=0.573$ \citep{Hunke:2008ly}. The snow is assumed to be fresh. The temperature profile of sea ice, $T_i$, is given by
\begin{equation}
\rho_i \cdot c_i\cdot \frac{\partial T_i}{\partial t}= \frac{\partial}{\partial z}\big(K_i \frac{\partial T_i}{\partial z}\big) - \frac{\partial I_{pen}}{\partial z},
\end{equation}
where $\rho_i$ is the sea ice density assumed to be constant and uniform, $c_i(T,S) $ is the specific heat of sea ice, $K_i(T,S)$ is the thermal conductivity of sea ice, $I_{pen}(z)$ is the solar penetrative radiation at depth $z$.

Melt is split into three processes based on the location of the melting: surface, bottom and lateral. Surface melting is calculated as:
\begin{align}
\Delta h &= (F_{0} - F_{ct})/q\\
F_{ct} &= K_h \cdot (T_{surf} - T_q),
\end{align}
where $\Delta h$ is the change in thickness, $F_{ct}$ is the heat conduction at the top of the ice, $q$ is the enthalpy of pure ice, $T_{surf}$ is the temperature at the surface of the ice. Bottom melting or growth is given by 
\begin{align}
\Delta h &= \frac{F_{cb} - F_{bot}}{q} \Delta t, \label{eq1}\\ 
F_{cb} &= K_h \cdot (T_q - T_{bot}), \label{eq2}\\
F_{bot} &= - \rho_w \cdot c_w \cdot c_h \cdot u_* \cdot (T_w - T_{fr}),\label{eq3}
\end{align}
where $F_{cb}$ is the conductive heat flux through the bottom of the ice, $q = -\rho_i(-c_0(\mu S+T)+L_0(1+\mu S/T)+c_w \mu S)$ is the enthalpy of the ice with $c_0=2106$ J/kg/K the specific heat of fresh ice at $0^\circ C$, $L_0 = 334 000$ J/kg the latent heat of fusion of fresh ice at $0^\circ C$ and $\mu=0.054$ K/psu the ratio between the freezing temperature and salinity of brine, $\Delta t$ is the time step, $K_h$ is the heat conductivity of sea ice, $T_q$ is the temperature of sea ice calculated from enthalpy $q$, $T_{bot}$ is the temperature at the base of the ice, $\rho_w$ is the water density, $c_w$ is the water heat capacity, $c_h$ is the heat exchange coefficient, $u_* = \sqrt{|\tau_w|/\rho_w}$ is the friction velocity, $|\tau_w|$ is the norm of the shear stress given by $\rho_w \sqrt{\frac{\partial u}{\partial y}^2+\frac{\partial v}{\partial x}^2}$, $T_w$ is the water temperature, $T_{fr}$ is the water freezing temperature. If $\Delta h$ is positive, basal growth is occurring. If $\Delta h$ is negative, bottom melting is occurring. Another process adds bottom ice: frazil ice. Frazil ice develops as supercooled droplets which form small crystals of ice in the mixed layer and, due to their buoyancy, reach the surface of the ocean. It is calculated as:
\begin{equation}\label{eqfraz}
frazil = \frac{(T_{fr} - T_w)\cdot c_w \cdot \rho_w \cdot h_{mix}}{q_0} 
\end{equation}
where $h_{mix}$ is the thickness of the ocean mixed layer, $q_0$ is the enthalpy of newly formed ice. Finally, lateral melting is calculated as 
\begin{align}
F_{side} &= rside \cdot E_{tot}, \label{eqml1}\\
rside &= \frac{m_1 \cdot (T_w - T_{bot})^{m_2} \cdot \pi}{\alpha \cdot f_D},\label{eqml2}
\end{align}
where $E_{tot}$ is the total energy available to melt ice and snow, $m_1 = 1.6\cdot 10^{-6}$ and $m_2 = 1.36$ are coming from \cite{MaykutPero1987}, $\alpha = 0.66$ from \cite{Steele1992}, $f_D$ is the flow diameter set at $300 \, m$. Note that the surface and bottom melt impact sea ice area when the sea ice thickness reach the thinnest thickness layer; half of the energy goes to reducing thickness while the other half goes to reducing sea ice area.

The main differences from the CSIM5 to CICE4 are an improved ridging scheme \citep{JGRC:JGRC10227}, a melt pond parameterization \citep{Holland:2011ly} and a new radiative transfer scheme \citep{Briegleb:2007ve} which includes effect of melt pond and absorber such as black carbon and dust on optical properties of sea ice. 

The seasonal cycle of the sea ice extent of the CCSM3 was too high during winter and slightly too high during summer \citep{holland2006c}. The CCSM4 solved both problems agreeing better with observations \citep{2012jahn}. The CCSM3 sea ice concentration is extending too far outside the Arctic Ocean while the CCSM4 contains it better \citep{gent2011}. The trend of sea ice extent between 1981 and 2005 is not well represented in the CCSM4 with 2 runs not simulating any significant loss. The CCSM4 has thicker ice close to Canada agreeing better with the observations than the CCSM3. The CCSM4 sea ice is too thick in the centre of the Arctic Ocean and it lacks  multiyear ice \citep{2012jahn}. The CCSM4 sea ice motion shows the standard large scale features such as the Beaufort Gyre and the Transpolar Drift Stream. Though, the simulated ice speeds are too large and show a wider distribution compared with observations.

\subsubsection{Future projections}\label{fs}

The CCSM simulates the present climate and projections of future climate up to 2099. The Intergovernmental Panel on Climate Change (IPCC) published studies of future projections of green house gases \citep{IPCC4,IPCC5}. Those green house gases reflect energy that would have normally escaped our atmosphere. The sum of all the extra energy from all the green house gases is called radiative forcing. The radiative forcings for the Special Report on Emissions Scenarios (SRES) \citep{IPCC4} and the Representative Concentration Pathways (RCP) \citep{IPCC5} are shown in figure \ref{SRES_RCP}. The SRES includes six families of scenario: A1FI, A1B, A1T, A2, B1, and B2. This thesis focus strictly on SRES A1B. The RCP includes four different pathways: RCP2.5, RCP 4.5, RCP6.0, RCP8.5. The scenarios RCP8.5 and RCP6.0 strictly increase while RCP4.5 stabilizes by 2070 and RCP2.5 peaks at $3\,W/m^2$ by 2040 and then decreases. Note that before 2060, RCP4.5 is stronger than RCP6.0. The same happens for the SRES A1B and A2. Before 2050, SRES A1B is stronger than A2 even if by 2100 A2 has a higher radiative forcing than A1B. Those periods of higher forcing for weaker scenarios can have an important impact on the climate. If an important event happens due to the higher radiative forcing during those years, the simulated climate could be significantly different. 

\begin{figure}
\center
\includegraphics[width=0.75\textwidth]{Processes/SRES_RCP.jpg}
\caption{Radiative forcing evolution of the different scenarios of the IPCC-AR4 (A1B, A2, B1) and IPCC-AR5 (RCPs).}
\label{SRES_RCP}
\end{figure}

\subsection{Conclusion}

The Arctic is the northern portion of planet Earth. At its center lies the Arctic Ocean covered by sea ice. Warm waters enter through the Bering Strait, east of Fram Strait as the West Spitzbergen Current, and through the Barents Sea. Cold waters exit through the Canadian Arctic Archipelago and west of Fram Strait  as the East Greenland Current. The Arctic Ocean is characterized by a cold halocline layer between $50\,m$ and $200\,m$ deep. The cold halocline layer insulates the surface ocean and its sea ice from the warm Atlantic layers $300\,m$ deep.

The sea ice is a complex material that under specific circonstances can reject its brine or keep it. The sea ice concentration of a defined region is measured by a satellite radiometer. The error on such measurements can reach up to $25\%$. Sea ice extent includes all area with more than $15\%$. It counteracts most of the error on sea ice concentration. Sea ice extent is minimum over the month of September. September sea ice extent reached new low records in 2002, 2005, 2007 and 2012. Sea ice thickness is complicated to measure and observations are scarce both spatially and temporally. It can be done with the help of Ultra Light Sonar, Electromagnetic Sounding, satellites or drilled holes. All observations point to a decrease in sea ice thickness. The sea ice thickness decreased comparably more than sea ice extent. 

The Arctic reacts more quickly and more intensely to climate changes. As the sea ice extent decreases, more open water is accessible to the solar radiation. While the sea ice reflects most of the solar radiation, the ocean absorbs most of it. The exposed water gets warmer and melt more sea ice which leaves more open water. It is a positive feedback named the Albedo feedback. It could thaw all summer sea ice in the next decades to come. 

\cite{ISI:000242942100008} observed abrupt reductions of Arctic September sea ice extent in the seven CCSM3 simulations forced by the SRES A1B leading to a summer ice free Arctic Ocean as soon as 2050. They stipulated that one to two years prior the abrupt loss of sea ice extent, pulses of ocean advective heat flux might have preconditioned the sea ice to a rapid loss. The thermodynamic interactions between the sea ice and the ocean is not well documented. It is usually calculated as a residual of all the sea ice forcings and the change in sea ice thickness. This thesis aims at describing the exchanges of heat between the sea ice and the ocean plus the sources of ocean heat reaching the sea ice. 

In order to proceed, the seven CCSM3 simulations forced by the SRES A1B will be studied. The CCSM3 has four modules - ocean, sea ice, land, atmosphere - and a coupler linking each parts. The ocean component is the POP model. It solves the Navier Stokes equations for a thin stratified fluid under hydrostatic equilibrium and Boussinesq approximation. The POP model uses a staggered Arakawa B grid with a displaced North Pole. The sea ice components is the CICE model. It has the same grid as the POP model. It solved the advection and forcings of sea ice. Its sea ice is elastic-viscous-plastic (EVP). One cell has five thickness categories plus open water. The CCSM3 has limited outputs making some analyses impossible. The CCSM4 is also studied for its extensive outputs. 

Chapter \ref{processes} investigates all the processes affecting sea ice focussing on the seven CCSM3 simulations forced by the SRES A1B. Chapter \ref{vert} presents how to calculate all the energy fluxes affecting the surface ocean layer. The CCSM4 having numerous outputs, the heat budget is more readily to compute. Chapter \ref{gates} shows an analysis of the advective heat fluxes through the gates of the Arctic Ocean. Both the CCSM versions 3 and 4 are compared. 




% Add a description of this thesis. Chapter 1 will treat blablabla...

% Several runs of the Community Climate System Model 3 (CCSM3) show rapid sea-ice decline \citep{Holland2006fa} similar in magnitude to the sea-ice decline that has been underway in the Arctic in the last decade. They showed that all rapid sea-ice declines are preceded by a pulse of heat from north Atlantic origin (see Figure \ref{fig:OHT}) followed by an increase in open water and an increased absorbed solar radiation in the surface ocean. This suggests that vertical ocean heat fluxes between the warm Atlantic layer and the surface Arctic mixed layer are playing an important role in sea-ice melt. This gives a new way to understand the recent warming of the North Atlantic drift and Atlantic layer in the Arctic \citep{Polyakov:2011bh}. However, the CCSM3 does not simulate a Cold Halocline Layer (CHL) and the Atlantic waters are at the correct depth but have a temperature bias of plus one degree celsius \citep{Tremblay:2007uq}. The bias could make the Arctic Ocean of the CCSM3 more vulnerable to the pulse of ocean heat of Atlantic origin. The CCSM4 has a poorly represented Atlantic layer \citep{Jahn:2011kx} which is too deep. Interestingly, the CCSM4 does not show rapid sea-ice decline which is inconsistent with recent observations of rapid decline of sea ice in the Arctic Ocean. Whether the impact of the ocean is indeed more important than we think or if there are other mechanisms that are not represented in current Global Climate Models (GCM) that would be responsible for the fast sea-ice retreat remains an open question.

%
%
%\cite{ISI:000242942100008} observed abrupt reductions of Arctic September sea ice extent in the seven CCSM3 simulations forced by the SRES A1B scenario. They defined an abrupt loss event when the time derivative of the five-year running September sea ice extent exceeds $-0.5$ million $km^2/y$. The spread of the abrupt loss event spans over every consecutive years around the event with a derivative over $-0.15$ million $km^2/y$ (paragraph [16]). They stated that thermodynamic processes were more impactful than the dynamic processes for the abrupt sea ice extent loss events without showing results supporting it (paragraph [10]). \cite{Holland2010} argued that the net ice formation (growth plus melt) is almost equal to transport on a yearly basis (table 1) showing that thermodynamic processes are not more important than dynamic processes contradicting their previous paper. 
%
%\cite{ISI:000242942100008} showed that the ocean heat transports through the gates of the Arctic Ocean increases with time (figure 3(a)). This ocean heat transport has "pulse-like" events that supposedly affect sea-ice one or two years later (figure 3(b)). Those abrupt reductions in Arctic September sea ice extent are considered robust since all of the seven simulations of the A1B scenario of the CCSM3 and numerous other projections exhibits abrupt transitions. 
%
%This paper received a lot of attention\footnote{Cited 359 times. Verified on \textit{https://apps.webofknowledge.com} May 26, 2017.}. The idea of an ice-free summer Arctic Ocean as soon as 2040 under a "middle of the road" scenario with nonlinear evolution of the sea ice extent was unheard of. The B1 scenario being optimistic with $550$ ppm CO2 by 2100, the pessimistic A2 scenario with $850$ ppm CO2 by 2100 and the reasonable A1B scenario with $720$ ppm CO2 \citep{SRES} (figure \ref{SRES_RCP}). Though the A1B scenario is between A2 and B1 by 2100, it is the most aggressive scenario before 2060. 
%\begin{figure}
%\center
%\includegraphics[width=0.75\textwidth]{Processes/SRES_RCP.jpg}
%\caption{Forcing evolution of the different scenarios of the IPCC-AR4 (A1B, A2, B1) and IPCC-AR5 (RCPs).}
%\label{SRES_RCP}
%\end{figure}
%
%The previous conception of sea ice decline was linear; the Arctic Ocean would lose sea ice at almost the same rate from fully covered to almost nothing. \cite{ISI:000242942100008} showed that it is not the case in some simulations by reveling the abrupt loss event. The CCSM3 simulated a reasonable sea ice extent compared to observations adducing a worthy Arctic model. The conclusion of the paper is: the rapid September sea ice extent losses in the CCSM3 A1B scenario were caused by an increasingly warmer atmosphere coupled with high interannual variability in heat entering the Arctic. 
%

% CCSM
%To address these questions, we will use the Community Climate System Model Version 3 and 4. We use this tool because CCSM3 is the model that gives the most realistic sea-ice decline, of all Coupled Model Intercomparison Project 3 (CMIP3) and 5 (CMIP5), consistent with the last three decades of  satellite observations. Nevertheless, CCSM and all other global climate models still underestimate the recent observed sea-ice decline in the Arctic \citep{GRL:GRL23061,GRL:GRL29477}. Furthermore this model suggests that the ocean may play a much larger role in the sea-ice mass balance than we previously thought. This is in line with recent measurements  (Slavin et al., in preparation) and preliminary high resolution ice-ocean models that show intense ocean heat fluxes along sea-ice leads \citep{McPhee:2005uq}. 
% Find the right Slavin paper.

%\addcontentsline{toc}{subsection}{References}
%\bibliographystyle{apa}
%\bibliography{Introduction.bib}


% Sea level

%Sea level rising threatens millions of lives \cite{SLR}. It is widely believed that floating ice does not contribute to sea level rising due to Archimedes law. Considering a piece of ice partly immersed in sea water with volume $V_i$, the submerged volume being $V_2$ and the dry volume being $V_1$. Once the ice is melted, the sea level will stay untouched if the volume of melted water is equal to the submerged volume. By the Archimedes law \citep{vallis}, the mass of the ice is equal to the displaced volume ($V_2$) times the density of the ocean, $\rho_o$,
%\begin{equation}\label{vol1}
%M_i \cdot g = V_2 \cdot \rho_o \cdot g.
%\end{equation}
%The same mass remains before and after the melt of the ice,
%\begin{equation}\label{vol2}
%M_{melt} = V_{melt} \cdot \rho_{melt} = M_i. 
%\end{equation}
%Using equation \ref{vol1} in \ref{vol2}, we obtain
%\begin{equation}
%V_{melt} = V_2 \cdot \frac{\rho_o}{\rho_{melt}}.
%\end{equation}
%If the density of the melted ice is the same as the ocean, the sea level stays unchanged. \cite{SLRice} showed that the densities are not the same and that a fully melted Arctic would increased the sea level by an extra $47 \, mm$ compared to $7 \, m$ for a fully melted Greenland and $60 \, m$ for a fully melted Antarctica estimated by \cite{GAslr}.

% INUIT HISTORY
%Under the liberal federal government of Louis Saint-Laurent during the 1950s, eight families from Inukjuak, Qu\'{e}bec were transported in the high Arctic to Grise Fiord on the souther tip of Ellesmere Island and to Resolute on Cornwallis Island (ADD FIGURE). In order for Canada to claim the Arctic Archipelago territory, it needed to prove a settlement of Canadians were living there. They were left without enough resources to feed themselves. The hunting ratio were more limited than in Qu\'{e}bec. They could not hunt enough land animals to feed everybody. Their only option was to hunt seals which proved to be very dangerous. The hunters needed to be close to the subzero water. Many of them fell and drowned. Exhausted and hungry, they stole food from the garbage bins of a military base not so far. When caught they were given infraction tickets. 
%
%The winter was harsher in the Arctic Archipelago than in northern Qu\'{e}bec where the snow is wet enough igloos early Winter. The Arctic snow was getting solid enough for igloos only by mid January where the temperature reached less than $-60^\circ C$. The inuits were unable to protect themselves from the cold. 
%
%By the end of the first winter, the inuits have been told that if they did not like their new home, the government would make sure they could come back to Inukjuak free of charge. Instead, they where forced to send letters to the rest of their families inviting them to join them. The letters were full of lies depicting the Arctic Archipelago as a paradise land. More people came. This was the High Arctic relocation. The following generation had several issues such as violence and alcoholism. 
%
%I collected those information from a survivor of the first wave. She was one year old when deported. What I described is not the version of the Government of Canada thought they apologized for the relocation in 2010 by the conservator federal government of Stephen Harper; John Duncan (Minister of Indian Affairs and Northern Development) stated: "The Government of Canada deeply regrets the mistakes and broken promises of this dark chapter of our history and apologizes for the High Arctic relocation having taken place. We would like to pay tribute to the relocatees for their perseverance and courage...The relocation of Inuit families to the High Arctic is a tragic chapter in Canada's history that we should not forget, but that we must acknowledge, learn from and teach our children. Acknowledging our shared history allows us to move forward in partnership and in a spirit of reconciliation."
%


