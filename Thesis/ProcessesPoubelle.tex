
% SHEBA
%
%On the 2$^{nd}$ of October 1997, the Canadian Coast Guard icebreaker Des Groseilliers stopped moving through the ice floes and allowed itself to be frozen in the ice pack. It drifted for a year until the 11$^{th}$ of October 1998 from 75$^/circ$N,142$^/circ$W to 80$^/circ$N, 162$^/circ$W. It's net displacement was nearly 800 km. During that year, the Surface HEat Budget of the Arctic (SHEBA) program took place. Between 20 and 50 scientists were present on the icebreaker at all time. They measured radiative fluxes, ocean-ice, ocean-atmosphere and atmosphere-ice heat fluxes, cloud properties, ice thickness, snow depth, ocean salinity, temperature and currents. 
%
%The expedition was motivated by the uncertainties of future projections by Global Climate Models (GCMs). Two key processes for the Arctic climate are ice-ocean albedo feedback and cloud-radiation feedback. Though well understood for 100 years, their impact differs greatly from GCMs to GCMs. The community needed year long measurements of Arctic features to refine their simulated projections and to understand the seasonal cycles of the Arctic climate. Evidences of significant changes in the Arctic alarmed the community. SHEBA was also a way to confirm and develop the worrying changes happening during the 1990s. 
%
%To do so, they used radar and LIDAR to measure properties and geometries of clouds, radiosonde to measure temperature, humidity and wind speeds up to 10 km, radiometers recording radiations located on the ground and aircraft, amphibious sleds to explore the changes in sea ice thickness, heat conduction, ice salinity, radiative properties of snow and ice. 
%
%They observed that the midwinter sky was overcast 40\% of the time while being continuously overcast during the summer. Under low clouds regime, the temperature was 10$^\circ C$ to 20$^\circ C$ warmer than under clear sky or high clouds regime. The net surface radiation was also higher from the trapped radiative energy by low clouds. Overall, the measurements at SHEBA concluded to a positive cloud feedback. 
%
%They observed that the sea ice thickness increased by 0.5m over 9 months. The sea ice melting period resulted in 0.64m of surface melt and 0.62m of bottom melt. They recorded 80 days of melt compared to an average of 55 days from previous measurements. 
%
%They measured the temperature of the ocean upper mixed layer being at freezing point during winter time and peaking at $0.3 ^\circ C$ late in July. During August a storm caused significant ice motion and ocean mixing. It resulted in an increased in bottom melt and a cooling of the ocean mixed layer. 
%
%Many more results were made to improved climate models. I only stated the most significant for this thesis. The results from SHEBA generated hundreds of scientific publications. Just before the SHEBA experiment, in 1996, an agreement between Vice President of the United States of America, Albert Gore, and Russia's Prime Minister, Viktor Chernomyrdin, was made to share Arctic measurements. Those measurements added to the ones from SHEBA helped corroborate trends.
%
%\cite{PT}, \cite{pero2003}, \cite{sheba2002}
%
%
%% OCEAN-ICE HEAT FLUXES 
%\citet{ISI:A1971I688400007} determined that a constant ocean heat flux of $2 \,Wm^{-2}$ through the whole year was necessary to reproduce the  measured sea-ice thickness evolution during a full seasonal cycle using a one dimensional thermodynamic sea-ice model forced with observed radiative and turbulent heat fluxes. They also showed that with an extra $8 \, Wm^{-2}$ of oceanic heat flux, the amount of ice melting in the summer is equal or higher than the amount of ice formation during winter, leading to an ice free summer. \citet{ISI:A1982NF38100017} measured ocean heat fluxes that are less than $2 \, Wm^{-2}$ from March to May north of the Fram strait. They deployed a thermocouple string and two thickness gauges on three sites 150 meters apart and calculated the oceanic heat fluxes from the difference between the conductive heat flux through sea ice, specific and latent heat fluxes at the ice base when ocean water transforms into sea ice. \citet{GRL:GRL17601} recorded an averaged oceanic heat flux of $2.6 \, Wm^{-2}$ with a buoy deployed close to the North Pole that drifted trough the Fram Strait. When the buoy reached the Yerkman Plateau, the oceanic heat flux reached values as high as $22 \, Wm^{-2}$. The sharp bathymetry and the absence of cold halocline layer led to a large ocean heat flux linked with tidal waves. \citet{McPhee:2005uq} showed that the heat flux along active Linear Kinematic Features (LKFs – leads) can be as large as $400 \, Wm^{-2}$. The mechanism invoked to explain such large heat fluxes is Ekman pumping associated with a positive curl in the surface ice-ocean stresses ventilating relic heat trapped beneath the mixed layer from previous summers.
%
%\citet{ISI:A1989AP36800003} calculated an averaged oceanic heat flux of $14 \, Wm^{-2}$ from a buoy moving through the Fram strait from the 14$^{th}$ of December 1987 to the 2$^{nd}$ of January 1988. By the end of December 1988, the buoy entered warm water and they measured an ocean heat flux of $128 \, Wm^{-2}$. \citet{4164498} calculated a strong seasonal cycle of the vertical ocean heat flux with a yearly average of $5.1 \, Wm^{-2}$ using the 1975 Arctic Ice Dynamics Joint EXperiment (AIDJEX). The maximum values of the vertical ocean heat flux was $40-60 \, Wm^{-2}$ in August and almost zero in winter. The experiment was composed of Conductivity-Temperature-Depth (CTD) profiles from four drifting stations. They also concluded that the energy was mainly coming from short wave radiation from the sun and a small portion was coming from warm deeper water. \citet{GRL:GRL15407} observed a strong seasonal cycle in the ocean heat flux with values of $2 \, Wm^{-2}$ during fall, winter and spring and values around $33 \, Wm^{-2}$ during the summer months when solar radiation enters the ocean's surface through leads and open water, causing basal ice melt. The measurements were done at  the Surface HEat Budget of the Arctic (SHEBA) experiment located North of Alaska in the Beaufort Sea and Chukchi Sea. They used an Ice-Mass-Balance (IMB) buoy and treated the oceanic heat flux as a residual following \citet{ISI:A1982NF38100017}. 
%
%\citet{Timmermans:2008fk} used Ice-Tethered Profilers (ITP) to determine the vertical ocean heat flux from the Atlantic waters to the surface in the Canada Basin. They calculated vertical ocean heat fluxes via double-diffusive staircase between $0.05 - 0.3 Wm^{-2}$. This is consistent with observed order of $10^{-6} m^2 s^{-1}$ of the diffusivity obtained from  measurements made in 2005 in the Canada Basin, Lamonosov Ridge and Amundsen Basin by \citet{GRL:GRL24453}. \cite{Lique:2013uq} obtained the same results from four deployed mooring as part of the Beaufort Gyre Observing System in the Canada basin during August 2003 and August 2011.
%
%
%
%%--------------------------------------------------------------------------------------------------------------------------------------------------------------------------
%%--------------------------------------------------------------------------------------------------------------------------------------------------------------------------
%%--------------------------------------------------------------------------------------------------------------------------------------------------------------------------
%%--------------------------------------------------------------------------------------------------------------------------------------------------------------------------
%
%
%
%
%
%
%
%
%
%
%\cite{ISI:000242942100008} observed abrupt reductions of Arctic September sea ice extent in the seven CCSM3 simulations forced by the SRES A1B scenario. They defined an abrupt loss event when the time derivative of the five-year running September sea ice extent exceeds $-0.5$ million $km^2/y$. The spread of the abrupt loss event spans over every consecutive years around the event with a derivative over $-0.15$ million $km^2/y$ (paragraph [16]). They stated that thermodynamic processes were more impactful than the dynamic processes for the abrupt sea ice extent loss events without showing results supporting it (paragraph [10]). \cite{Holland2010} argued that the net ice formation (growth plus melt) is almost equal to transport on a yearly basis (table 1) showing that thermodynamic processes are not more important than dynamic processes contradicting their previous paper. 
%
%\cite{ISI:000242942100008} showed that the ocean heat transports through the gates of the Arctic Ocean increases with time (figure 3(a)). This ocean heat transport has "pulse-like" events that supposedly affect sea-ice one or two years later (figure 3(b)). Those abrupt reductions in Arctic September sea ice extent are considered robust since all of the seven simulations of the A1B scenario of the CCSM3 and numerous other projections exhibits abrupt transitions. 
%
%This paper received a lot of attention\footnote{Cited 359 times. Verified on \textit{https://apps.webofknowledge.com} May 26, 2017.}. The idea of an ice-free summer Arctic Ocean as soon as 2040 under a "middle of the road" scenario with nonlinear evolution of the sea ice extent was unheard of. The B1 scenario being optimistic with $550$ ppm CO2 by 2100, the pessimistic A2 scenario with $850$ ppm CO2 by 2100 and the reasonable A1B scenario with $720$ ppm CO2 \citep{SRES} (figure \ref{SRES_RCP}). Though the A1B scenario is between A2 and B1 by 2100, it is the most aggressive scenario before 2060. 
%\begin{figure}
%\center
%\includegraphics[width=0.75\textwidth]{Processes/SRES_RCP.jpg}
%\caption{Forcing evolution of the different scenarios of the IPCC-AR4 (A1B, A2, B1) and IPCC-AR5 (RCPs).}
%\label{SRES_RCP}
%\end{figure}
%
%The previous conception of sea ice decline was linear; the Arctic Ocean would lose sea ice at almost the same rate from fully covered to almost nothing. \cite{ISI:000242942100008} showed that it is not the case in some simulations by reveling the abrupt loss event. The CCSM3 simulated a reasonable sea ice extent compared to observations adducing a worthy Arctic model. The conclusion of the paper is: the rapid September sea ice extent losses in the CCSM3 A1B scenario were caused by an increasingly warmer atmosphere coupled with high interannual variability in heat entering the Arctic. 
%
%As ground breaking \cite{ISI:000242942100008} was, some points are ambiguous. Most of the argument is based on the presence of rapid sea ice loss events in all seven A1B simulation and a multitude of other simulations from other climate models. It is unclear how robust are the loss events on modifications of the event criteria. It is evident that if the time derivative of the five-year running mean threshold of $-0.5$ million $km^2/y$ is increased , more events would arise since the required slope would be more gentle. On the other side, if the time derivative threshold is decreased, there would be less event since the required slope would be steeper. Similarly, if the spread criteria of $-0.15$ million $km^2/y$ is increased, the events would be longer and if decreased, the events would be shorter. Working with time derivative of a running mean is the insidious part. 
%
%The sea ice extent value of a five-year running mean is given by:
%\begin{equation}
%SIE_i^5 = \frac{SIE_{i-2}+SIE_{i-1}+SIE_{i}+SIE_{i+1}+SIE_{i+2}}{5}.  
%\end{equation}
%Using this definition, the time derivative of the five-year running mean is given by:
%\begin{equation}
%\frac{dSIE_i^5}{dt} = \frac{SIE_{i+1}^5 - SIE_{i}^5}{\Delta t} = \frac{SIE_{i+3} - SIE_{i-2}}{5 \Delta t},
%\end{equation}
%which is the slope between the sea ice extent three years after the year of interest and the sea ice extent two years before. A very unorthodox way to describe what happens at some time because it is non-symmetrical. Changing the extent of the running mean could bring unforeseen differences in the number and spread of the abrupt September loss events as shown in table \ref{criteria}. The only convincing event happening in simulation b.ES01 during 2027 to 2030. Without surprises, the authors as well as the community never used that criterion again. 
%
%\begin{table}
%\center{
%\begin{tabular}{c | c | c | c | c | c }
%  & 1 yrm & 3 yrm & 5 yrm & 7 yrm & 9 yrm \\
%  \hline 
%a & 22 events & 2006, 2035 & 2036 & - & - \\
%  & 1913 to 2047 & 2039, 2048& & & \\
%  \hline
%b.ES01 & 14 events & 1999, 2025-26,  & 2027, 2030 & 2028-29 & 2028-29 \\
%  & 1920 to 2043 & 2031-32, 2037 & & & \\
%  \hline
%c & 20 events & 1989, 2032 & 2032 & - & - \\
%  & 1900 to 2041 & 2042 & & & \\
%  \hline  
%e & 25 events & 12 events & 2032, 2043 & 2031 & - \\
%  & 1910 to 2056 & 2011 - 2057 & & & \\
%  \hline  
%f.ES01 & 18 events & 8 events & 2013-14, 2045 & - & - \\
%  & 1914 to 2044 & 1957 - 2044& & & \\
%  \hline
%g.ES01 & 16 events & 2011, 2046-47 & 2044 & - & - \\
%  & 1904 to 2037 & & & & \\
%\end{tabular}
%}
%\caption{Abrupt events loss of sea ice for different values of year running mean (yrm). For high number of events, the table indicates the first and last year with an event.}
%\label{criteria}
%\end{table}
%
%Direr than the cryptic criteria of abrupt sea ice loss is the obscure fate of sea ice. Sea-ice extent does not represent adequately the state of the ice. It is defined as the sum of all grid cell areas (A) with more than $15\%$ sea ice concentration (SIC) over the Northern-Hemisphere,
%\begin{equation}
%SIE = \sum^{NH} A(SIC>15\%).
%\end{equation}
%Any cell with $15\%$ or more SIC counts fully while cells with less than $15\%$ SIC are left out. This definition changes depending on the institution calculating it with the criterion ranging between $15\%$ to $30\%$. 


The real physical state of sea ice is given by the sea ice area (SIA)
\begin{equation}
SIA = SIC \cdot A
\end{equation}
and sea ice volume (SIV)
\begin{equation}
SIV = SIC\cdot h \cdot A,
\end{equation}
where $h$ is the average ice thickness of the cell. The thickness can be calculated by dividing the sea ice volume by the sea ice area. Those real physical ice states are not studied in \cite{ISI:000242942100008}.
% p12 of CICE 5.1

Those two thoughts experiment show how unrelated sea ice extent can be from real sea ice measures. If all the sea ice of the Arctic Ocean could be compacted over one region, the sea ice area and volume would be unchanged but the sea ice extent would be lower. The sea ice extent can vary without any change in sea ice volume or area. The opposite is also true; the sea ice volume and sea ice area can change without altering the sea ice extent. If all the sea ice melts to half a meter and all the cells get $15\%$ covered, the sea ice volume and area would drastically decrease while the sea ice extent would stay the same. Major sea ice transformation can occur without changing the sea ice extent.  

Sea ice extent becomes important while comparing with satellite measurements which are known to be inaccurate. When cross-referenced with measurements from other satellites and  monthly averaged, scientists become confident in the results of the sea ice extent; not sea ice area. Satellites cannot differentiate between ice pond and ocean bringing higher errors during the summer. They cannot make adequate measurements when the weather is cloudy - the clouds are seen as ocean - or too windy making the ocean be seen as ice and  coastal ice can be observed as sea ice \citep{Cavalieri1995}. The National Snow and Ice Data Center (NSIDC) suggests not using daily sea ice concentration from satellite observations.\footnote{https://nsidc.org/data/docs/noaa/g02135\_seaice\_index/\#acquisition\_proc, section 6}. 

Even if the differences between sea ice extent, sea ice area, sea ice thickness, sea ice volume and their physical representations are well known in the community, they do not act on it. Most model studies are still fully focused on sea ice extent without regard to sea ice area or volume. We wish to offer deeper and more physical insight on how to analyze model output by providing an example. 

From the analysis displayed in \cite{ISI:000242942100008}, the real fate of the sea ice is eluded in terms of thickness/volume and area. When working with observations, it can happen that some measurements are impossible or unreliable and are left out of the analysis. It is not the case with numerical models who calculate and can store every prescribed process.  This thesis intend to demystify the physical cause(s) of the rapid September sea ice extent decline.  


\subsection{Sea ice extent, area and volume} \label{SIEvsSIA}

As stated before, a change in sea ice extent does not imply a change in sea ice. Figure \ref{cover} shows the yearly maximum (March or winter) and minimum (September or summer) Arctic sea ice area superposed on the September sea ice extent. 
\begin{figure}
\center
\noindent\includegraphics[width=0.85\linewidth]{Processes/aice.jpg}
\caption{March and September SIA (blue) and September SIE (red). The years with a rapid sea ice decline are identified with a black vertical line.}
\label{cover}
\end{figure}
The September sea ice extent starts to decrease rapidly after year 2000 when the forcing from the A1B scenario is turned on. The September sea ice area follows a similar evolution to the September sea ice extent but with a lower value. As expected, when the sea-ice extent reaches very low values, the sea ice area do so as well. During abrupt loss of September sea ice extent, usually the decrease in September sea ice area is more significant but is not always the case as in simulation g.ES01. 

The winter sea ice area starts to decrease slowly after the Arctic Ocean is ice free in September. This loss is mainly located at the entrance of the Barents Sea opening (figure \ref{march2099}). There is also a small retreat at the entrance of the Bering Sea in simulations a, b.ES01 and e. 
\begin{figure}
\center
\noindent\includegraphics[width=0.85\linewidth]{Processes/MarchAice2099.jpg}
\caption{March 2099 sea ice area.}
\label{march2099}
\end{figure}

The yearly maximum (winter) and minimum (summer) Arctic sea ice volume starts decreasing  1975, 25 years before the sea ice area and extent (figure \ref{icevol}). The summer Arctic sea ice volume reaches almost zero by 2050 while the winter sea ice volume stabilizes around $10 \cdot 1000 \, km^3$. During the abrupt loss of September sea ice extent, the volume decreases slightly more rapidly but nothing unobserved between 1975 and the event. 
\begin{figure}
\center
\noindent\includegraphics[width=0.85\linewidth]{Processes/volume2.jpg}
\caption{Yearly maximum and minimum sea ice volume ($m^3$).}
\label{icevol}
\end{figure}

From 1975 till the end of the simulations, sea ice volume decreases steadily and rapidly. After 25 years of rapid volume loss, the sea ice becomes thin enough to melt completely and affect the sea ice area. This loss in sea ice area consequently diminishes the sea ice extent. After 50 years of decreasing sea ice volume and area, the ice gets thin over a broader region (\textcolor{green}{figures \ref{1m} and \ref{BAaice}}) and the sea ice area of several regions gets close to the sea-ice extent threshold of $15\%$ causing the abrupt loss event described in \cite{ISI:000242942100008}.

\begin{figure}
\center
\noindent\includegraphics[width=0.85\linewidth]{Processes/BandA1m.jpg}
\caption{One-meter sea ice thickness contour before (blue) and after (red) rapid loss event.}
\label{1m}
\end{figure}

\begin{figure}
\center
\noindent\includegraphics[width=0.85\linewidth]{Processes/BandAaice.jpg}
\caption{Sea ice concentration difference before and after rapid loss event.}
\label{BAaice}
\end{figure}

\begin{figure}
\center
\noindent\includegraphics[width=0.85\linewidth]{Processes/SIEminusSICinSIE.jpg}
\caption{Difference between the SIE and the sea ice area inside the SIE zone.}
%\label{}
\end{figure}

\begin{figure}
\center
\noindent\includegraphics[width=0.85\linewidth]{Processes/CoverPourcentageIceExtent.jpg}
\caption{Percentage of the SIE zone covered with sea ice.}
%\label{}
\end{figure}

\begin{figure}
\center
\noindent\includegraphics[width=0.85\linewidth]{Processes/aiceANDivol.jpg}
\caption{September sea ice area (blue) and September sea ice volume (green).}
%\label{}
\end{figure}

% \begin{figure}
% \center
% \noindent\includegraphics[width=0.85\linewidth]{Processes/aiceVSivol.jpg}
% \caption{Sea ice area in function of sea ice volume. Each color represents a month.}
% \label{}
% \end{figure}

% \begin{figure}
% \center
% \noindent\includegraphics[width=0.85\linewidth]{Processes/aiceVSivolY.jpg}
% \caption{Sea ice area in function of sea ice volume. Each color represents a year.}
% \label{}
% \end{figure}

Therefore the loss of sea ice extent is linked to loss of sea ice area and volume. In the next sections, we will study the processes affecting sea ice area and volume: compaction and dispersion (section \ref{compact}), ridging and lead opening (section \ref{ridge}), transport out of the Arctic Ocean (section \ref{transportsec}), sea ice formation (section \ref{forma}) and sea ice melt (section \ref{melta}). In order to find the guilty processes of the rapid loss of sea ice extent, we build our case by contradiction. To obtain a change in the slope of the sea ice area and volume, one or many of the processes must break the old equilibrium. If a process does not change during the rapid sea ice extent loss, it cannot cause it. Any steady or decreasing field during the rapid loss events is discarded as a cause. 





\subsubsection{Dynamic VS Thermodynamic}

\cite{ISI:000242942100008} stated that the dynamic processes play little role in rapid loss of sea ice extent (paragraph [10]). The standard output of the CCSM3 includes four variables quantifying the strength of the dynamic and thermodynamic impact on SIA and SIV: daidtd for the dynamic changes in SIA, daidtt for the thermodynamic changes in SIA, dvidtd for the dynamic changes in SIV, dvidtt for the thermodynamic changes in SIV. Thermodynamic and dynamic impacts the SIA and SIV at almost the same weight (figure \ref{daidt} and \ref{dvidt}). The thermodynamic and dynamic impact on SIA is stable until 2020 where it starts to decrease. The SIA tendency variability increases from 2000 until 2099. The thermodynamic variability increases also after the year 2000 while the dynamic variability stays stable. Hence, the relative importance of the dynamic processes decreases in terms of inter-annual variability.

The variability of the thermodynamic and dynamic changes in SIV and its tendency are high compared to their relative distance. The three curves meet by 2060. It is now clear that dynamic processes do not bear a little role on the evolution of sea ice volume and area and therefore, on sea ice extent.

\begin{figure}
\center
\noindent\includegraphics[width=0.85\linewidth]{Processes/TvsD_aice.jpg}
\caption{Thermodynamic changes, dynamic changes and tendency of SIA. Each color represents a different simulation.}
\label{daidt}
\end{figure}

\begin{figure}
\center
\noindent\includegraphics[width=0.85\linewidth]{Processes/TvsD_aiceVol.jpg}
\caption{Thermodynamic changes, dynamic changes and tendency of SIV. Each color represents a different simulation.}
\label{dvidt}
\end{figure}



The dynamic processes include compaction, ridging and transport out of the Arctic Ocean. The thermodynamic processes include melt and ice creation.

\subsubsection{Compaction/Dispersion}\label{compact}

Compaction means higher sea ice concentration over a lower number of cells. For instance, 10 cells with $60\%$ sea ice coverage that, under compaction, shrink to only 6 cells at $100\%$ coverage and 4 remaining cells at $0\%$ coverage. In the physical world, if the sea ice concentration of a cell becomes low and the wind pushes the ice toward a hard surface (coast or highly concentrated sea ice), the ice will be compacted. This process only affects the sea ice concentration distribution, which in turn affects the sea ice extent leaving the sea ice volume, thickness and area untouched. 

The opposite of compaction is dispersion. The wind can also push further away low concentrated sea ice cells increasing the sea ice extent. The CCSM3 does not offer a direct way to quantify the compaction and dispersion in their standard output. We have to stop our analysis of this process and hope that compaction cancels dispersion or can be easily discarded compared to important processes.

\subsubsection{Ridging/Lead opening}\label{ridge}

Ridging happens inside a cell in contrast with compaction which happens between cells. Ridging rarely occurs in cells with low sea ice concentration because it is easier to move or compact sea ice than break it and pile it up. Ridging happens in high sea ice-concentrated cells by increasing the local thickness and diminishing the sea ice concentration under strong  wind stress. For instance, a cell completely covered with 1 meter-thick sea ice could ridge to a cell with 2 meter-thick sea ice with $50\%$ coverage. Note that the mean thickness of the cell stays unchanged. This process rarely takes the highly concentrated cells under 15\% and does not affect the sea ice extent. Though sea ice area is affected by ridging.

In the physical world, ridging occurs over broad ice region. The loss in area will occur further away where the ice is not tied. In the model, the lost area is located in the cell and the resulting open water can freeze again increasing the thermodynamic ice creation. 

The opposite of ridging is leads opening. If the ice is anchored to the shore or the ocean floor and the wind pushes away from it hard enough, the sea ice breaks creating a lead of open water. In this case, the sea ice area and volume are unchanged while the sea ice extent can increase. Just as compaction and dispersion, the CCSM3 standard output does not allow a direct analysis of ridging and lead creation. Again, we must hope those processes cancel or are small compared to the important processes. 

\subsubsection{Transport out of the Arctic Ocean}\label{transportsec}

The sea ice volume transport (SIVT) at the gates of the Arctic Ocean (figure \ref{AA}) is calculated as the product of the component of the sea-ice velocity component perpendicular to the gates ($u_\perp$) and the sea-ice thickness, summed over the width of the gate,
\begin{equation}
SIVT = \sum^{gate} u_\perp \cdot h.
\end{equation}
It is not an exact result since the variables in the product are monthly averaged.
\begin{equation}
u_\perp \cdot h = (\overline{u}_\perp+ u_\perp') \cdot (\overline{h}+h') = \overline{u}_\perp \cdot \overline{h} + u_\perp'\cdot \overline{h} + \overline{u}_\perp \cdot h' + u_\perp' \cdot h'
\end{equation}
\cite{Arfeuille} showed that the mixed terms, $u_\perp'\cdot \overline{h}$ and $\overline{u_\perp} \cdot h'$ are significant for the Fram Strait export of sea ice while the term $u_\perp' \cdot h'$ is negligible. 

In the case of the six A1B simulations of the CCSM3, the error between the calculated transport and a variable integrating every dynamic change in volume (dvidtd) is small  (figure \ref{VerifDyn}) excepted for simulation c over the $21^{st}$ century. Other than that the error decreases in time from $0.5 \, 1000 \, km^3/y$ down to $0.2 \, 1000 \, km^3/y$ with the dynamic changes in volume always higher than the calculated transport out of the Arctic. At the beginning of the simulations the error represents $12\%$ while it represents $30\%$ by 2100. This error is not troublesome since, as will be demonstrated in section \ref{discussion}, the transport is a small term compared to the other processes and its error being even smaller does not change any of our conclusions. It also sways that the compaction/dispersion and ridging/lead opening processes are not important in those simulations.
\begin{figure}
\center
\noindent\includegraphics[width=0.85\linewidth]{Processes/VerifDyn.jpg}
\caption{Calculated yearly sea ice transport out of the Arctic Ocean (blue), yearly changes in sea ice volume caused by dynamic processes (green), error between the two (red).}
\label{VerifDyn}
\end{figure}

The Arctic sea ice export through the Fram Strait is dominating all the other gates (figure \ref{export}); $4\cdot \, 1000 \,km^3/y$ for the Fram Strait, $0.2\cdot 1000\, km^3/y$ for the Barents sea opening, $0.04\cdot  \, 1000 \,km^3/y$ for the CAA, $-0.2\cdot  \, 1000 \,km^3$ for the Bering Strait for the 1900s. The export through the Fram Strait decreases linearly, between $-0.007 \cdot  \, 1000 \,km^3/y^2$ and $-0.02 \cdot \, 1000 \,km^3/y^2$, until the A1B scenario starts year 2000 where it decreases exponentially. The exponent value ranging from $-0.017 \, y^{-1}$ to $-0.019 \, y^{-1}$, until 2080 where it stabilizes around $0.7 \cdot 1000 \, km^3$. 

\begin{figure}
\center
\noindent\includegraphics[width=0.85\linewidth]{Processes/Export.jpg}
\caption{Fram (blue), Barents (green), CAA (red) and Bering (teal) ice transport out of the Arctic Ocean.}
\label{export}
\end{figure}



\subsection{Processes}
\subsubsection{Sea ice formation}\label{forma}

Sea ice formation occurs through basal ice formation and frazil ice formation. As the latent heat of solidification is conducted upwards from the ocean's surface through to the sea ice to the cold atmosphere, basal ice form at the base of the ice. The model computes it as:
\begin{align}
\Delta h &= \frac{F_{cb} - F_{bot}}{q} \Delta t, \label{eq1}\\ 
F_{cond} &= K_h \cdot (T_q - T_{bot}), \label{eq2}\\
F_{bot} &= - \rho_w \cdot c_w \cdot c_h \cdot u_* \cdot (T_w - T_{fr}),\label{eq3}
\end{align}
where $\Delta h$ is the change in thickness, $F_{cb}$ is the conductive heat flux through the bottom of the ice, $F_{bot}$ is the turbulent heat flux between ocean and ice, $q$ is the enthalpy of the ice, $\Delta t$ is the time step, $K_h$ is the heat conductivity of sea ice, $T_q$ is the temperature of sea ice calculated from enthalpy, $T_{bot}$ is the temperature at the base of the ice, $\rho_w$ is the water density, $c_w$ is the water heat capacity, $c_h$ is the heat exchange coefficient, $u_* = \sqrt{|\tau_w|/\rho_w}$ is the friction velocity, $|\tau_w|$ is the norm of the shear stress given by $\rho_w \sqrt{\frac{\partial u}{\partial y}^2+\frac{\partial v}{\partial x}^2}$, $T_w$ is the water temperature, $T_{fr}$ is the water freezing temperature (\cite{Hunke:2008ly}). If $\Delta h$ is positive, basal growth is occurring. If $\Delta h$ is negative, bottom melting is occurring.  

Frazil ice develops as supercooled droplets which form small crystals of ice in the mixed layer and, due to their buoyancy, reach the surface of the ocean. It is calculated as:
\begin{equation}\label{eqfraz}
frazil = \frac{(T_{fr} - T_w)\cdot c_w \cdot \rho_w \cdot h_{mix}}{q_0} 
\end{equation}
where $h_{mix}$ is the thickness of the ocean mixed layer, $q_0$ is the enthalpy of newly formed ice. 

Basal ice growth is always more important than the frazil ice growth (figure \ref{growth}), respectively $10.1 \cdot 1000 \, km^3/y$ and $2.1 \cdot 1000 \, km^3/y$ averaged over the 20th century. The basal ice growth is stable until the rapid sea ice decline and decreases briskly, ranging between $-0.038 \cdot 1000 \, km^3/y^2$ and $-0.051 \cdot 1000 \, km^3/y^2$, until 2080 where it stabilizes at $6.8 \cdot 1000 \, km^3/y$. 

The standard output of the CCSM3 does not allow a precise calculation of the conductive heat flux. The heat conductivity of sea ice is not stored as well as the enthalpy and the temperature derived from it. It is impossible for us to pinpoint what changed in the ice or the ocean modifying the basal ice growth after the rapid decline. 

\begin{figure}
\center
\noindent\includegraphics[width=0.85\linewidth]{Processes/Growth2.jpg}
\caption{Frazil ice formation (blue) and basal ice growth (green) averaged over the Arctic domain in figure \ref{AA}.}
\label{growth}
\end{figure}

\subsubsection{Sea ice melt}\label{melta}

Melt is split into three processes based on the location of the melting: surface, bottom and lateral. Surface melting is calculated as:
\begin{align}
\Delta h &= (F_{surf} - F_{ct})/q \label{eqm1}\\
F_{ct} &= K_h \cdot (T_{surf} - T_q), \label{eqm2}
\end{align}
where $F_{surf}$ is the sum of all the surface fluxes, $F_{ct}$ is the heat conduction at the top of the ice, $T_{surf}$ is the temperature at the surface of the ice (\cite{Hunke:2008ly}).

Bottom melting is calculated using equations \ref{eq1}, \ref{eq2} and \ref{eq3} when the result is negative. 

Lateral melting is calculated as 
\begin{align}
F_{side} &= rside \cdot E_{tot}, \label{eqml1}\\
rside &= \frac{m_1 \cdot (T_w - T_{bot})^{m_2} \cdot \pi}{\alpha \cdot f_D},\label{eqml2}
\end{align}
where $E_{tot}$ is the total energy available to melt ice and snow, $m_1 = 1.6\cdot 10^{-6}$ and $m_2 = 1.36$ are coming from \cite{MaykutPero1987}, $\alpha = 0.66$ from \cite{Steele1992}, $f_D$ is the flow diameter set at $300 \, m$. 

The basal melt increases nonlinearly from $3.9 \cdot 1000 \, km^3/y$ on average during the 1900s up to $6.7 \cdot 1000 \, km^3/y$ on average during the rapid loss events followed by a decrease ending at $4.8 \cdot 1000 \, km^3/y$ (figure \ref{melt}). The surface melt increases steadily from $2.7 \cdot 1000 \, km^3/y$ in the 1900s up to $4.5 \cdot 1000 \, km^3/y$ during the rapid loss events and after decreases ending at $2.4 \cdot 1000 \, km^3/y$. The lateral melting decreases slightly during the full duration of the simulation at  $-0.4 \cdot km^3/y^2$ starting at $0.8 \cdot 1000 \, km^3/y$ in the 1900s. 

Just as for the ice creation, the analysis is stopped by the limited output from the CCSM3. The analysis of the basal and surface melting is limited by the missing heat conductivity in the ice, the enthalpy and the temperature derived from it just as for the basal growth. The total energy available for melting involved in the calculation of the lateral melting is not present in the standard output. Our analysis must stop here if we wish to avoid unconvincing results.

Basal and surface melt increase until the rapid decline and then decrease. The amount of energy available to melt strictly increases since the atmosphere and ocean temperature strictly increase. Though, after the rapid ice loss there is less sea ice to receive this energy thus a decreasing amount of energy melting sea ice. 

\begin{figure}
\center
\noindent\includegraphics[width=0.85\linewidth]{Processes/Melt2.jpg}
\caption{Basal (blue), lateral (green) and surface (red) melt averaged over the Arctic domain in figure \ref{AA}.}
\label{melt}
\end{figure}





\subsection{Discussion}\label{discussion}


\subsubsection{Ocean, surface and transport}

Based on equations \ref{eqm1}, \ref{eqm2}, \ref{eqml1} and \ref{eqml2}, melt is increased by stronger surface fluxes, warmer atmosphere and warmer ocean. From equations \ref{eq1}, \ref{eq2}, \ref{eq3} and \ref{eqfraz}, formation is diminished by those factors. It is thus logical to group together the processes mainly driven by surface fluxes on one side and the ocean driven processes on the other. 

The surface fluxes increase the surface temperature melting sea ice directly and indirectly through a warmer surface. The surface melt belongs to this category. The surface ocean temperature is the main driver of bottom melt, basal growth, frazil formation and lateral melt. In both categories, the conduction term is a consequence of temperature changes at the bottom and the top of the sea ice. It cannot drive melt or creation; it is a passive process. The energy must come from an active source such as surface fluxes or the ocean.  For the bottom melt/formation, the friction velocity can modify the amount of energy transferred but is not a source of energy that can melt. It has the role of a catalyst or a blocker. For those reasons, it does not induce melt and should not be considered as an active source of energy. 

We decided to put the lateral melting in the ocean driven category even if it depends on the total available melting energy. Even if debatable, since the lateral melt is such a small term in those simulations of the CCSM3, its category does not change our conclusions. 

The last category includes all the sea ice transport. It deserves its own category since it does not depend on energy fluxes but on ice presence and ice velocity. Simply, we divide every process in surface, bottom and transport. 

The sum of the ocean driven processes is decreasing steadily from 2000 to 2080 at a rate  of $-0.04 \cdot 1000 \, km^3/y^2$ (figure \ref{procG}). It stabilizes between 2080 and 2100 at $3.3 \cdot 1000 \, km^3/y$. Before 2000, simulations a and e show a decrease in the sum of the ocean driven processes. Simulations f.ES01 and g.ES01 show an increase between 1900 and 1950 and then a decrease. Simulations b.ES01 and c slightly decrease between 1900 and 2000. The three situations show different solutions of the model for the fight between ice creation and melting at the base of the sea ice.

The surface processes strength increases, $0.007 \cdot 1000 \, km^3/y^2$, melting more sea ice until the rapid sea ice decline. After the rapid loss, it decreases, $-0.02 \cdot 1000 \, km^3/y^2$, until 2080 and stabilizes at $2.4 \cdot 1000 \, km^3/y$ for the rest of the simulation. 

The sea ice transport out of the Arctic drastically decreases from $4 \cdot 1000 \, km^3/y$ to $2 \cdot 1000 \, km^3/y$ during the abrupt loss of sea ice extent and then decreases down to $0.8 \cdot 1000 \, km^3/y$ between 2080 and 2100 where it stabilizes. The variability of the sea ice transport decreases just as remarkably with a standard deviation of $1000 \, km^3/y$ for the first 50 simulated years down to $0.3 \cdot 1000 \, km^3/y$.

\begin{figure}
\center
\noindent\includegraphics[width=0.85\linewidth]{Processes/Proc.jpg}
\caption{Surface processes (blue), ocean driven processes (green), sea ice transport out of the Arctic Ocean (red). To follow easily the evolution of each processes' category, positive surface processes represents loss of sea ice, positive ocean driven processes represents sea ice formation, positive transport represents loss of sea ice.}
\label{procG}
\end{figure}

While the total ice creation at the bottom of sea ice is decreasing and the surface processes are melting more ice both working together toward the rapid sea ice loss, the sea ice transport decreases slowing down the sea ice loss. Therefore, it cannot be an important source for the rapid sea ice loss. 

The bottom processes are driven by the sea surface temperature. The sea surface temperature increases exponentially and the rapid sea ice declines happen at $-0.9^\circ C$ (figure \ref{SSTg1}). Over the full simulation, the sea surface temperature increased by up to $8^\circ C$ in the Barents Sea (figure \ref{SSTg2}). The same warm region in the Barents Sea is found in the figure \ref{march2099} as missing sea ice in the Arctic Ocean during the month of March. It is very difficult to melt sea ice during winter time when the atmosphere is extremely cold. The fact that the ocean can stop the sea ice formation through the whole winter shows the strength the ocean has on sea ice in the CCSM3. Sadly, for unknown reason it is impossible to access the sea surface temperature of simulation a. 

\begin{figure}
\center
\noindent\includegraphics[width=0.85\linewidth]{Processes/SST.jpg}
\caption{Sea surface temperature averaged over the Arctic Ocean.}
\label{SSTg1}
\end{figure}

\begin{figure}
\center
\noindent\includegraphics[width=0.85\linewidth]{Processes/SSTspacial.jpg}
\caption{Sea surface temperature difference between the 2090s and the 1900s.}
\label{SSTg2}
\end{figure}

The surface processes are driven by the surface temperature and the surface fluxes. The surface temperature is increasing steadily with time (figure \ref{Tsfcg1}). The rapid loss of sea ice extent happens when the surface temperature reaches $10^\circ C$. From 1900 to 2100, the sea-ice surface temperature increases mainly right at the center of the Arctic Ocean close to the Barents Sea opening. Both the ocean and the atmosphere worked together weakening specifically this region in those simulations of the CCSM3. 

\begin{figure}
\center
\noindent\includegraphics[width=0.85\linewidth]{Processes/Tsfc.jpg}
\caption{Sea-ice surface temperature averaged over the Arctic Ocean.}
\label{Tsfcg1}
\end{figure}

\begin{figure}
\center
\noindent\includegraphics[width=0.85\linewidth]{Processes/TsfcSpacial.jpg}
\caption{Sea-ice surface temperature difference between the 2090s and the 1900s.}
\label{Tsfcg2}
\end{figure}

The surfaces fluxes of importance are short wave and long wave. We are interested in the fluxes absorbed by sea ice. The long wave fluxes decreases slightly until the A1B scenario is turned on in 2000 and then starts decreasing rapidly until the abrupt loss of sea ice where it also abruptly decreases (figure \ref{SurfIce}). The short wave is less influence by the changes in sea ice area. It decreases slowly and steadily during whole simulation. The resulting surface flux decreases everywhere over the Arctic Ocean but in the Barents Sea and at the junction of the Central Arctic Ocean and the Barents sea where it increases. 

\begin{figure}
\center
\noindent\includegraphics[width=0.85\linewidth]{Processes/SurfIce.jpg}
\caption{Short wave absorbed by sea ice (blue), longwave absorbed by sea ice (green) and longwave emitted (red) averaged over the Arctic Ocean.}
\label{SurfIce}
\end{figure}

\begin{figure}
\center
\noindent\includegraphics[width=0.85\linewidth]{Processes/SurfIceSpatial.jpg}
\caption{Total radiation (absorbed plus emitted) difference between the 2090s and the 1900s at the surface of sea ice in $W/m^2$.}
\label{SurfIce2}
\end{figure}

In those simulations of the CCSM3, the Barents Sea is a region of very high intensity and everything is aligned to melt sea ice. No other Arctic region is remotely similar. We find surprising that this region receives more energy from surface radiation that the rest of the Arctic Ocean. We expected a more constant distribution. Unfortunately, the standard output of the CCSM3 does not allow us to break down every radiative fluxes to the variable used calculating them. We cannot explain why it is the way it is. 



% Bottom interaction is driven by the ocean surface temperature. 

% Bottom growth and melt are calculated with the same equation and are driven by ocean temperature,
% \begin{align}
% Bot &=  K_h \cdot (T_q - T_{bot}) + \rho_w \cdot c_w \cdot c_h \cdot u_* \cdot (T_w - T_{fr})
% 	&\propto T_w.
% \end{align}
% The conduction is a consequence of temperature difference. It cannot be the driver of important changes. It adapts to new situation but does not force climate changes such as the rapid loss events. The same argument holds for the surface melt,
% \begin{align}
% Top &= F_{surf} - K_h \cdot (T_{surf} - T_q)\\
% 	&\propto F_{surf}.
% \end{align}
% The lateral melt is driven by sea surface temperature just as the bottom melt and growth,
% \begin{align}
% Lat &= \frac{m_1 \cdot (T_w - T_{bot})^{m_2} \cdot \pi}{\alpha \cdot f_D} \cdot E_{tot},\\
% 	&\propto (T_w - T_{bot})^{1.36} \cdot (T_w + F_{surf}) \cdot Ice \, Presence.
% \end{align}
% The ice transport depends on ice thickness, concentration and wind.

% There are three major driver to any ice changes: sea surface temperature, surface forcings and ice transport. Anything happening at the bottom of sea ice is driven by sea surface temperature. Top melting is driven by surface forcing. The lateral melting is a mix of all the drivers. Being small compared to the energy transfers happening at the bottom and top of sea ice, it is easily discarded. 



% Splitting processes into dynamic and thermodynamic does not help understand the relative importance of processes. It is important to group together the processes sharing 

% One might think that even a weak lateral melt could be important because it directly affects sea ice area enabling albedo feedback. If it was the case, we would see an increasing lateral melt coming from more available lateral ice surface and increased surface temperature. It would also follow closely the loss of sea ice area; when the sea ice area decreases more rapidly, the lateral melt increases more rapidly. It is not what is simulated since it slightly decrease with time. Therefore, the lateral melt is ruled out leaving bottom and surface melt as main cause of the abrupt loss of sea ice. In the CCSM, the surface and bottom melt impact sea ice area when the sea ice thickness reach the thinnest thickness layer, $60\,cm$ for the simulations of interest. Half of the energy goes to reducing thickness while the other half goes to reducing sea ice area. 


% \begin{figure}
% \center
% \noindent\includegraphics[width=0.85\linewidth]{Processes/IceProc2.jpg}
% \caption{Sea ice formation (blue), melt (green) and transport (red) averaged over the Arctic domain in figure \ref{AA}.}
% \label{proc}
% \end{figure}

% Sea-ice formation is always more important than sea ice melt (figure \ref{proc}). Sea ice formation is quite steady during the $20^{th}$ century with a mean value of $12.3\cdot 10^12 \, m^3/year$ (80\% basal and 20\% frazil ice formation) and a yearly variability of $1.6\cdot 10^12 \, m^3/year$. Sea ice melt, on the other hand, gains strength over the same period at a rate of $0.02 \cdot 10^12 \, m^3/year^2$ from $7.6 \cdot 10^12 \, m^3/year$ to $9.6 \cdot 10^12 \, m^3/year$. Reading only these terms, it appears the Arctic sea ice is growing. To complete the picture, one should include the sea ice transport out of the Arctic Ocean. Over the $21^{st}$ century when the forcing scenario is turned on, the formation and melt reach the same rate at $8.3 \cdot 10^12 \, m^3/year$ with a difference of $0.4 \cdot  10^12 \, m^3/year$ by 2099 with the formation always being larger than the melt. During the $21^{st}$ century, almost all of the newly formed sea ice melts. We conclude that the rapid summer sea ice declines are not generated from the sea ice formation because it is steady. On the other hand, sea ice melt increases rapidly until it reaches a threshold. Hence, sea ice melt must trigger the rapid sea ice decline.  



%As stated before, a change in sea ice extent does not imply a change in sea ice. Figure \ref{cover} shows the yearly maximum (March or winter) and minimum (September or summer) Arctic sea ice area superposed on the September sea ice extent. 
%\begin{figure}
%\center
%\noindent\includegraphics[width=0.85\linewidth]{Processes/aice.jpg}
%\caption{March and September SIA (blue) and September SIE (red). The years with a rapid sea ice decline are identified with a black vertical line.}
%\label{cover}
%\end{figure}
%The September sea ice extent starts to decrease rapidly after year 2000 when the forcing from the A1B scenario is turned on. The September sea ice area follows a similar evolution to the September sea ice extent but with a lower value. As expected, when the sea-ice extent reaches very low values, the sea ice area do so as well. During abrupt loss of September sea ice extent, usually the decrease in September sea ice area is more significant but is not always the case as in simulation g.ES01. 









%
%Therefore, the surface melt is driven by the surface temperature and the radiative fluxes. The surface temperature is increasing steadily with time (figure \ref{Tsfcg1}). The rapid loss of sea ice extent happens when the annualOrSeptember surface temperature reaches $-10^\circ C$. From 1900 to 2100, the sea-ice surface temperature increases mainly right at the center of the Arctic Ocean close to the Barents Sea opening. 
%%Both the ocean and the atmosphere worked together weakening specifically this region in those simulations of the CCSM3. 
%
%\begin{figure}
%\center
%\noindent\includegraphics[width=0.85\linewidth]{Processes/Tsfc.jpg}
%\caption{Sea-ice surface temperature averaged over the Arctic Ocean.}
%\label{Tsfcg1}
%\end{figure}
%
%\begin{figure}
%\center
%\noindent\includegraphics[width=0.85\linewidth]{Processes/TsfcSpacial.jpg}
%\caption{Sea-ice surface temperature difference between the 2090s and the 1900s.}
%\label{Tsfcg2}
%\end{figure}
%
%The surfaces fluxes of importance are short wave and long wave. We are interested in the fluxes absorbed by sea ice. The long wave fluxes decreases slightly until the A1B scenario is turned on in 2000 and then starts decreasing rapidly until the abrupt loss of sea ice where it also abruptly decreases (figure \ref{SurfIce}). The short wave is less influence by the changes in sea ice area. It decreases slowly and steadily during whole simulation. The resulting surface flux decreases everywhere over the Arctic Ocean but in the Barents Sea and at the junction of the Central Arctic Ocean and the Barents sea where it increases. 
%
%\begin{figure}
%\center
%\noindent\includegraphics[width=0.85\linewidth]{Processes/SurfIce.jpg}
%\caption{Short wave absorbed by sea ice (blue), longwave absorbed by sea ice (green) and longwave emitted (red) averaged over the Arctic Ocean.}
%\label{SurfIce}
%\end{figure}
%
%\begin{figure}
%\center
%\noindent\includegraphics[width=0.85\linewidth]{Processes/SurfIceSpatial.jpg}
%\caption{Total radiation (absorbed plus emitted) difference between the 2090s and the 1900s at the surface of sea ice in $W/m^2$.}
%\label{SurfIce2}
%\end{figure}
%


% The bottom processes are driven by the sea surface temperature. The sea surface temperature increases exponentially and the rapid sea ice declines happen at $-0.9^\circ C$ (figure \ref{SSTg1}). Over the full simulation, the sea surface temperature increased by up to $8^\circ C$ in the Barents Sea (figure \ref{SSTg2}). The same warm region in the Barents Sea is found in the figure \ref{march2099} as missing sea ice in the Arctic Ocean during the month of March. It is very difficult to melt sea ice during winter time when the atmosphere is extremely cold. The fact that the ocean can stop the sea ice formation through the whole winter shows the strength the ocean has on sea ice in the CCSM3. Sadly, for unknown reason it is impossible to access the sea surface temperature of simulation a. 

%\begin{figure}
%\center
%\noindent\includegraphics[width=0.85\linewidth]{Processes/SSTspacial.jpg}
%\caption{Sea surface temperature difference between the 2090s and the 1900s.}
%\label{SSTg2}
%\end{figure}




% In those simulations of the CCSM3, the Barents Sea is a region of very high intensity and everything is aligned to melt sea ice. No other Arctic region is remotely similar. We find surprising that this region receives more energy from surface radiation that the rest of the Arctic Ocean. We expected a more constant distribution. Unfortunately, the standard output of the CCSM3 does not allow us to break down every radiative fluxes to the variable used calculating them. We cannot explain why it is the way it is. 
% Not sure about this paragraph...




%\subsubsection{Energy sources}
%
%Several fluxes are linked to the same energy source, for example: the atmosphere, the ocean or solar radiation. To investigate the source of the sea ice volume decrease, it is imperative to group processes on their energy source. Based on equations \ref{eqm1}, \ref{eqm2}, \ref{eqml1} and \ref{eqml2}, melt is increased by stronger surface fluxes, warmer atmosphere and warmer ocean. From equations \ref{eq1}, \ref{eq2}, \ref{eq3} and \ref{eqfraz}, formation is diminished by those factors. It is thus logical to group together the processes mainly driven by surface fluxes on one side and the ocean driven processes on the other. 
%
%The surface fluxes increase the surface temperature melting sea ice directly and indirectly through a warmer surface. The surface melt belongs to this category. The surface ocean temperature is the main driver of bottom melt, basal growth, frazil formation and lateral melt. In both categories, the conduction term is a consequence of temperature changes at the bottom and the top of the sea ice. It cannot drive melt or creation; it is a passive process. The energy must come from an active source such as surface fluxes or the ocean.  For the bottom melt/formation, the friction velocity can modify the amount of energy transferred but is not a source of energy that can melt. It has the role of a catalyst or a blocker. For those reasons, it does not induce melt and should not be considered as an active source of energy. 
%
%We decided to put the lateral melting in the ocean driven category even if it depends on the total available melting energy. Even if debatable, since the lateral melt is such a small term in those simulations of the CCSM3, its category does not change our conclusions. 
%
%The last category includes all the sea ice transport. It deserves its own category since it does not depend on energy fluxes but on ice presence and ice velocity. Simply, we divide every process in surface, bottom and transport. 
%
%We end up with three energy sources: surface (\ref{Es}), ocean (\ref{Eo}) and transport (\ref{Et}) that are depicted in their respective sections.
%
%\subsubsection{Surface}\label{Es}
%
%The surface processes strength increases, $0.007 \cdot 1000 \, km^3/y^2$, melting more sea ice until the rapid sea ice decline. After the rapid loss, it decreases, $-0.02 \cdot 1000 \, km^3/y^2$, until 2080 and stabilizes at $2.4 \cdot 1000 \, km^3/y$ for the rest of the simulation. 
%
%
%The surface processes are driven by the surface temperature and the surface fluxes. The surface temperature is increasing steadily with time (figure \ref{Tsfcg1}). The rapid loss of sea ice extent happens when the surface temperature reaches $10^\circ C$. From 1900 to 2100, the sea-ice surface temperature increases mainly right at the center of the Arctic Ocean close to the Barents Sea opening. Both the ocean and the atmosphere worked together weakening specifically this region in those simulations of the CCSM3. 
%
%\begin{figure}
%\center
%\noindent\includegraphics[width=0.85\linewidth]{Processes/Tsfc.jpg}
%\caption{Sea-ice surface temperature averaged over the Arctic Ocean.}
%\label{Tsfcg1}
%\end{figure}
%
%\begin{figure}
%\center
%\noindent\includegraphics[width=0.85\linewidth]{Processes/TsfcSpacial.jpg}
%\caption{Sea-ice surface temperature difference between the 2090s and the 1900s.}
%\label{Tsfcg2}
%\end{figure}
%
%The surfaces fluxes of importance are short wave and long wave. We are interested in the fluxes absorbed by sea ice. The long wave fluxes decreases slightly until the A1B scenario is turned on in 2000 and then starts decreasing rapidly until the abrupt loss of sea ice where it also abruptly decreases (figure \ref{SurfIce}). The short wave is less influence by the changes in sea ice area. It decreases slowly and steadily during whole simulation. The resulting surface flux decreases everywhere over the Arctic Ocean but in the Barents Sea and at the junction of the Central Arctic Ocean and the Barents sea where it increases. 
%
%\begin{figure}
%\center
%\noindent\includegraphics[width=0.85\linewidth]{Processes/SurfIce.jpg}
%\caption{Short wave absorbed by sea ice (blue), longwave absorbed by sea ice (green) and longwave emitted (red) averaged over the Arctic Ocean.}
%\label{SurfIce}
%\end{figure}
%
%\begin{figure}
%\center
%\noindent\includegraphics[width=0.85\linewidth]{Processes/SurfIceSpatial.jpg}
%\caption{Total radiation (absorbed plus emitted) difference between the 2090s and the 1900s at the surface of sea ice in $W/m^2$.}
%\label{SurfIce2}
%\end{figure}
%
%
%
%\subsubsection{Ocean}\label{Eo}
%
%The sum of the ocean driven processes is decreasing steadily from 2000 to 2080 at a rate  of $-0.04 \cdot 1000 \, km^3/y^2$ (figure \ref{procG}). It stabilizes between 2080 and 2100 at $3.3 \cdot 1000 \, km^3/y$. Before 2000, simulations a and e show a decrease in the sum of the ocean driven processes. Simulations f.ES01 and g.ES01 show an increase between 1900 and 1950 and then a decrease. Simulations b.ES01 and c slightly decrease between 1900 and 2000. The three situations show different solutions of the model for the fight between ice creation and melting at the base of the sea ice.
%
%The bottom processes are driven by the sea surface temperature. The sea surface temperature increases exponentially and the rapid sea ice declines happen at $-0.9^\circ C$ (figure \ref{SSTg1}). Over the full simulation, the sea surface temperature increased by up to $8^\circ C$ in the Barents Sea (figure \ref{SSTg2}). The same warm region in the Barents Sea is found in the figure \ref{march2099} as missing sea ice in the Arctic Ocean during the month of March. It is very difficult to melt sea ice during winter time when the atmosphere is extremely cold. The fact that the ocean can stop the sea ice formation through the whole winter shows the strength the ocean has on sea ice in the CCSM3. Sadly, for unknown reason it is impossible to access the sea surface temperature of simulation a. 
%
%\begin{figure}
%\center
%\noindent\includegraphics[width=0.85\linewidth]{Processes/SST.jpg}
%\caption{Sea surface temperature averaged over the Arctic Ocean.}
%\label{SSTg1}
%\end{figure}
%
%\begin{figure}
%\center
%\noindent\includegraphics[width=0.85\linewidth]{Processes/SSTspacial.jpg}
%\caption{Sea surface temperature difference between the 2090s and the 1900s.}
%\label{SSTg2}
%\end{figure}
%
%In those simulations of the CCSM3, the Barents Sea is a region of very high intensity and everything is aligned to melt sea ice. No other Arctic region is remotely similar. We find surprising that this region receives more energy from surface radiation that the rest of the Arctic Ocean. We expected a more constant distribution. Unfortunately, the standard output of the CCSM3 does not allow us to break down every radiative fluxes to the variable used calculating them. We cannot explain why it is the way it is. 
%% Not sure about this paragraph...
%
%\subsubsection{Transport}\label{Et}
%
%
%The sea ice transport out of the Arctic drastically decreases from $4 \cdot 1000 \, km^3/y$ to $2 \cdot 1000 \, km^3/y$ during the abrupt loss of sea ice extent and then decreases down to $0.8 \cdot 1000 \, km^3/y$ between 2080 and 2100 where it stabilizes. The variability of the sea ice transport decreases just as remarkably with a standard deviation of $1000 \, km^3/y$ for the first 50 simulated years down to $0.3 \cdot 1000 \, km^3/y$.
%
%\begin{figure}
%\center
%\noindent\includegraphics[width=0.85\linewidth]{Processes/Proc.jpg}
%\caption{Surface processes (blue), ocean driven processes (green), sea ice transport out of the Arctic Ocean (red). To follow easily the evolution of each processes' category, positive surface processes represents loss of sea ice, positive ocean driven processes represents sea ice formation, positive transport represents loss of sea ice.}
%\label{procG}
%\end{figure}
%
%While the total ice creation at the bottom of sea ice is decreasing and the surface processes are melting more ice both working together toward the rapid sea ice loss, the sea ice transport decreases slowing down the sea ice loss. Therefore, it cannot be an important source for the rapid sea ice loss. 
%








% Cite others research of sea ice heat budget and tell how different is your work. 






% In order to pinpoint the source of the rapid summer sea ice loss events, we started by looking at the dynamic processes: compaction and ridging. Compaction and ridging processes require strong wind events and do not affect the sea ice volume. We observed a decrease in mean wind forcing and a stable maximum wind forcing. In addition, there is a decrease in ice volume correlated to the sea-ice extent loss. We concluded that compaction and ridging were not the sources of rapid sea ice loss. 

% Transport out of the Arctic Ocean is negligible compared to the thermodynamic processes. The melt and formation of sea ice are both important, but only the melt increases significantly between the year 2000 and time of the rapid loss events. The melt occurs mainly at the bottom of the sea ice and increases the most compared to top melt. Lateral melt is negligible. Bottom melt is driven by the ice-ocean turbulent heat flux which increases everywhere in the Arctic Ocean. However, the increase is strongest over the peripheral seas ($30 W/m^2$ in 100 years) than in the central Arctic ($7 W/m^2$ in 100 years). During the winter, a high ice-ocean turbulent heat flux bulges through the Barents Sea Opening. 

% Ice-ocean turbulent heat flux is calculated from friction velocity and surface temperature. The friction velocity is not always maximum during the rapid summer sea ice extent loss while the temperature always reaches a new maximum. The abrupt loss events happens when the surface temperature is around $-1^\circ C$. It allows us to conclude that the main source of rapid summer sea ice decline is a warmer ocean surface.

% The main factor influencing the ocean surface temperature are: surface forcing, heat fluxes through the gates of the Arctic Ocean and ocean vertical heat fluxes. Each one of those processes will be studied in the following sections: ocean vertical heat fluxes in section \ref{vert}, surface fluxes in section \ref{rad} and gates energy transfer in section \ref{gates}.

%\addcontentsline{toc}{subsection}{References}
%\bibliographystyle{plainnat}
%\bibliography{Processes.bib}














