\section{Introduction}

The Arctic is undergoing a rapid and critical transformation. Sea ice has been decreasing both in extent and thickness since 1979, the first year we have been able to monitor it in 1979 \citep{overland2013}. The Arctic has lost $50\%$ of its sea ice extent and $75\%$ of its sea ice volume showing no signs of slowing or stopping. All wildlife living on sea ice must adapt forcing a major mutation of the ecosystems \citep{Pfirman2009pb,Laidre07122008}. The marine ecosystems are also adjusting to the new warmer waters entering the Arctic Ocean \citep{Beszc2011,wassmann2011}.  The Inuit infrastructure and health are vulnerable to climate changes. Harvesting is now more dangerous and access to hunting areas is increasingly difficult \citep{ford2006,ford2009}. There is also the threat of permafrost melt releasing trapped green house gases enhancing global warming \citep{permafrost}. The only bright side is the opening of new, quicker, marine routes \citep{Johannessen:2000uq}. Those new routes could lead to a decrease in fuel consumption in comparison with the usual longer paths.  

\cite{ISI:000242942100008} presented the results of six simulations under the Special Report on Emission Scenario (SRES) A1B from the Community Climate System Model version 3 (CCSM3). The simulations are characterized by a summer ice-free Arctic Ocean as early as 2040 preceded by a rapid loss of September sea ice extent. \cite{ISI:000242942100008} hypothesize that the ocean could catalyze the rapid loss of sea ice after observing pulse of warm water entering the Arctic Ocean one or two years prior the rapid losses. The ocean heat transfer to the sea ice has been theorized to be $2-4\,W/m^2$ constant throughout the year \citep{ISI:A1971I688400007}. Measurements showed that the thermal interactions between the ocean and the sea ice follow a seasonal cycle and that it differs greatly spatially \citep{ISI:A1982NF38100017,GRL:GRL17601,ISI:A1989AP36800003,GRL:GRL15407}. 

This thesis aims at describing the exchanges of heat between the sea ice and the ocean, and the sources of ocean heat. More specifically, it focuses on the six simulations of the CCSM3 under the SRES A1B presented in \cite{ISI:000242942100008}. Chapter \ref{thearctic} presents a literature review of the Arctic Ocean and its sea ice. It also presents the CCSM3 and its features such as its grids and model components. Chapter \ref{processes} investigates all the processes affecting the sea ice with a more acute interest for the interactions between the Arctic Ocean and its sea ice. Since only the first layer of the Arctic Ocean interacts with the sea ice, chapter \ref{vert} presents all the sources of heat affecting the top ocean layer attempting its energy-temperature budget. Chapter \ref{gates} studies the ocean advective heat fluxes through the gateways of the Arctic Ocean for both the CCSM version 3 and version 4. 





%The Arctic is facing a dire future. Sea ice shows a contant decrease in extent and thickness for as long as we can monitor it from  satellite measurements in 1979. The tremendous loss of sea ice has detrimentally affected the Arctic wild life and the first nations living there. The threat of permafrost melt releasing trapped green house gases enhancing global warming. Warmer ocean water enters the Arctic Ocean which is slowly damaging the marine ecosystem. 



%This thesis is dedicated to verify the hypothesis that ocean heat transport is an important source of melting that could trigger rapid sea ice loss. The first chapter focus on the energy budget of sea ice for the same simulations presented in \cite{ISI:000242942100008}. The second chapter describes how the heat budget of the ocean is done in the CCSM3 and CCSM4. The third chapter characterizes the ocean heat transport through each gates of the Arctic Ocean for the CCSM3 and CCSM4 followed by the conclusion. 

%A
%- ?to verify? -> ?to verifying? ou ?to the verification of?
%- ?the first chapter focuses?
%- ?same simulations? -> ?simulations?; ici, le same laisse croire que dans le texte tu as déjà parlé de ces simulations.
%- ?each gate? sans s 
%- Ici le lecteur ne sait pas ce que sont CCSM3 et CCSM4. Cela sera précisé plus tard, mais il serait quand même souhaitable que tu dises un mot pour dire que ce sont des versions du code.
%- Est-ce que cette introduction est incomplète? Pour un doctorat, six lignes d?introduction me semblent bien peu!