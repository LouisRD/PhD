\section*{Statement of Originality}
\addcontentsline{toc}{section}{Statement of Originality}
The following elements of the thesis show original scholarship and represent distinct
contributions to knowledge:
\begin{itemize}
\item \cite{ISI:000242942100008} observed rapid loss of September sea ice extent in all of the CCSM3 SRES A1B simulations. They notice that the ocean heat transport to the Arctic precedes the mean Arctic sea ice thickness by two years. The work presented in chapter \ref{processes}, investigates the role of all the sources of melt, growth and transport for the CCSM3 simulations under the SRES A1B. The ocean heat transport considered in \cite{ISI:000242942100008} includes all heat transport at $55^\circ$ North. The region north of the  $55^{th}$ parallel encloses more than just the Arctic. This thesis works with an original region enclosing the Arctic Ocean tightly in the hope of a better correlation between the heat transport  and the sea ice. \cite{Holland2010} analyzed the total melt, total growth and total transport of several GCMs and reanalysis including the CCSM3. This thesis investigates further by distinguishing (1) the total melt from its locations (bottom, top and lateral), (2)  the sea ice creation from its two processes (basal and frazil), (3) the sea ice transport through the gates of the Arctic Ocean (the Fram Strait, the Barents Sea Opening, the CAA and the Bering Strait). In addition, inspired by the work of \cite{winton2011} which observed linear trends between the Arctic sea ice extent and the global temperature in the GFDL simulations, a decreasing exponential relationship between the sea ice volume and the extra forcing from the SRES A1B scenario is presented. 


\item Chapter \ref{vert} started off as a study of vertical heat fluxes of the CCSM3 simulations. Since the vertical diffusion is not part of the standard output, Professor Bruno Tremblay hired an NCAR employee, Laura Landrum with the help of Marika Holland, to provide a code calculating vertical heat diffusion as a residual of the temperature-heat equation. The code was finished when I started my PhD. As a test of the code, I used all the heat components included in their code and tried to reconstruct the temperature-heat equation. This work proved to be incredibly more complicated than I anticipated. Chapter \ref{vert} and the appendices of this thesis are the results of this work. While the reconstruction of the temperature/heat equation is not original in itself - it is already known that the CCSM solves it -, it is a formidable tool to possess. No indications on how it should be done are published. Many scholars I met would have greatly appreciated the presence of such a tool for their own research. 


\item Despite the inconclusive results of chapter \ref{vert}, I pushed forward and analyzed the heat transport through the gateways of the Arctic Ocean of the CCSM versions 3 and 4. Heat transports are mainly recorded over the Fram Strait and the Bering Strait \citep{Beszc2011} because their locations are well accepted. The two other gateways are the Canadian Arctic Archipelago and the Barents Sea Opening. The water flowing through the many openings of the Canadian Arctic Archipelago is arduous to document. The Barents Sea Opening is not clearly defined. It spans from the east side of Svalbard unto Europe. It can end anywhere from the northern part of Norway to the eastern part of Novaya Zemlya Island. Considerable amount of sea ice exits the Arctic through the Fram Strait \citep{smeds2017} making it an excellent location to study. The Bering Strait is shallow and narrow.  \cite{Woodgate2010} showed that the heat transport through the Bering Strait has an important impact on the sea ice thickness and cover in the proximity of the strait. Even though the heat transports have a notable impact on the Arctic sea ice, it is not thoroughly studied utilizing GCMs results. The freshwater transports are studied more extensively \citep{fw2001}, from observations data\citep{fwobs} and from GCMs results \citep{fwmodels}, since they have a considerable effect on the global thermohaline circulation. Chapter \ref{gates} tackles the neglect of the model community toward the heat transport through the gateways of the Arctic Ocean. The results from the CCSM3 and the CCSM4 hold significant discrepancies. 




\end{itemize}














