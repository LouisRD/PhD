\section*{Résumé}
\addcontentsline{toc}{section}{Résumé}

Cette thèse explore les interactions thermodynamiques entre l'océan Arctique et sa glace de mer. Pour ce faire, les résultats des six simulations du CCSM3 soumises au futur scénario SRES A1B sont étudiés. \cite{ISI:000242942100008} ont ont constaté que les pertes abruptes de superficie de glace en septembre observées ces simulations sont précédées par des maximums de transport de chaleur océanique une ou deux années au paravent. L'analyse présentée ici pousse plus loin en considérant tous les mécanismes physiques influençant la glace de mer en connexion avec les années de perte abrupte de glace. La fonte de glace s'accentue jusqu'aux pertes abruptes de glace tandis que la création de glace stagne et que l'expulsion de glace diminue. La fonte de glace est donc responsable des pertes abruptes de superficie de glace. La fonte à la base de la glace est le procédé de fonte le plus important et est celui qui augmente le plus. En moyenne, l'océan transfert à la glace $20\,W/m^2$ en 1900 et $120\,W/m^2$ en 2100. Ces flux de chaleur sont considérablement plus élevés que celui prédit par \cite{ISI:A1971I688400007} de $2\,W/m^2$. Cette hausse de transfert thermique est générée par la  hausse exponentielle de la température de surface de l'océan Arctique. Les sources de chaleur qui influencent la température de surface de l'océan Arctique peuvent être étudiées en complétant le bilan énergique océanique. Ce bilan ne peut pas être complété sans une erreur substantielle de $50\,W/m^2$ même en utilisant des variables manquantes grâce aux résultats du CCSM4. Néanmoins, l'étude des flux de chaleur à travers les divers accès à l'océan Arctique est possible et permet d'identifier l'impact de ces flux de chaleurs sur l'océan Arctique et sa glace de mer. Cinq simulations du CCSM4 sous le scénario RCP 6.0 sont ajoutées aux six simulations précédemment nommées. Pour le CCSM3, le flux de chaleur total à l'océan Arctique est dominé par le flux passant par l'ouverture de la mer de Barents. Le flux de chaleur total augmente la température océanique de l'Arctique à force de $35\,W/m^2$ en 1900 et $130\,W/m^2$ en 2100. Pour le CCSM4, les flux de chaleur passant par toutes les voies d'accès sont importants. Le flux de chaleur total du CCSM4 réchauffe l'océan Arctique à force de $19\,W/m^2$ en 1900 et $60\,W/m^2$ en 2100. Le CCSM3 prévoit un réchauffement océanique, causé par les flux advectifs de chaleur, deux fois plus fort que celui du CCSM4. 








