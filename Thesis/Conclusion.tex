\section{Conclusion}

This thesis studied the thermal interactions between the Arctic sea ice and the Arctic Ocean. In addition, the sources of ocean heat influencing the surface layer of the Arctic Ocean has been analyzed. Chapter \ref{processes} investigated the evolution of the sea ice volume for the six simulations of the CCSM3 under the SRES A1B described in \cite{Holland2006fa}. Sea ice volume starts decreasing in 1950 while the sea ice extent decrease starts in the $21^{st}$ century; $40\%$ of the sea ice volume has been lost between 1950 and 2000 compared to only $8\%$ of the sea ice extent. The sea ice volume decreases exponentially in function of the extra forcing from the future emission scenario with a half life of $1\,W/m^2$. Half of the remaining volume will be lost when the extra forcing increases of $1\,W/m^2$. The transport of sea ice volume out of the Arctic Ocean happens almost entirely through the Fram Strait. In 1900, $4\cdot 1000\,km^3/y$ exits the Arctic Ocean compared to $1000\,km^3/y$ in 2100. The most important and most growing thermodynamic process is basal melt. It is determined by the amount of heat from the ocean transferred to the sea ice. The amount of heat transferred is $30\,W/m^2$ by 1900 and ascend up to $130\,W/m^2$ by 2100. This utmost increase is caused by the exponential increase in surface temperature. The surface temperature starts at $-1.5^\circ C$ in 1900 ending at $0^\circ C$ in 2100. The rapid loss of September sea ice extent described in \cite{Holland2006fa} cannot be explained solely by the bottom melt though. All the other processes are important and cannot be neglected. 

Chapter \ref{vert} investigated all the heat sources influencing the Arctic Ocean surface temperature for the CCSM3. Unfortunately, the vertical heat diffusion is not output. It is possible to evaluate it as a residual of the temperature change and the other heat sources. To ascertain the inclusion of all the other heat sources, closing a column budget is necessary. The column budget has zero vertical diffusion at the ocean's surface and at the ocean's floor. This exercise leads to an error of $\pm 2\,W/m^2$ over the Global Ocean, $\pm 3\,W/m^2$ over the Arctic Ocean and $\pm 150\,W/m^2$ over the columns. The Arctic Ocean being well stratified, only weak diffusive fluxes, less than $1\,W/m^2$, are expected \citep{Timmermans:2008fk}. An error or $\pm 150\,W/m^2$ is unacceptable. The error can be lowered using instantaneous temperature snapshots for the temporal derivative instead of monthly averaged temperature fields. The CCSM3 does not provide such fields but the CCSM4 does. The CCSM4 temperature-heat budget closes up to $\pm 0.4\,W/m^2$ over the Global Ocean and the Arctic Ocean. Even if the Global and Arctic Oceans' budget error improved appreciably, the heat budget of the columns has an error of $\pm 50\,W/m^2$. This unacceptable error does not bring enough confidence in our calculation to trust the heat budget presented in this thesis. Verifying every term in the CCSM code would bring the most assertive confidence in the heat budget. Disappointingly, even if shedding some light on several heat sources, it did not point to any refinement of the temperature-heat budget calculation.

Instead of examining all the heat sources as a whole, chapter \ref{gates} scrutinize only one source: the advective heat fluxes through the gateways of the Arctic Ocean. The gateways include the Fram Strait, the Barents Sea opening, the Canadian Arctic Archipelago and the Bering Strait. There are considerable discrepancies between the CCSM3 and CCSM4 forecasts. Over the $20^{th}$ century, the heat transports through the Fram Strait for the CCSM3 and CCSM4 are similar averaging at $9\,TW$ yearly. They diverge over the $21^{st}$ with the CCSM3 decreasing down to $-25\,TW$ and the CCSM4 increasing up to $15\,TW$. Through the Barents Sea Opening, the heat transport increased from $25\,TW$ up to $150\,TW$ for the CCSM3 while it increased from $0\,TW$ yearly up to $30\,TW$ for the CCSM4. Over the Canadian Arctic Archipelago, the heat transport decreases modestly for both versions of the CCSM. The CCSM3 predicts a heat transport of $2\,TW$ in 1900 and $1\,TW$ in 2100 while the CCSM4 predicts $10\,TW$ in 1900 and $8\,TW$ in 2100. The results for the Bering Strait are similar over the versions 3 and 4 of the CCSM starting close to $-1\,TW$ in 1900 and ending at $8\,TW$ in 2100. Advective heat fluxes are elusive and different calculations could bring different results. Their impact on warming or cooling the Arctic Ocean can only be ascertained by doing a full budget over all the gates. In both versions of the CCSM, the total heat transport is positive, warming the Arctic Ocean. The CCSM3 total heat transport is dominated by the Barents Sea Opening. It starts at $35\,TW$ and ends at $130\,TW$. The CCSM4 total heat transport receives notable contributions from all the gateways. It begins at $19\,TW$ in 1900 and stops at $60\,TW$. The advective heat transport is twice more important in the CCSM3 than in the CCSM4. The CCSM heat transport forecasts do not agree well with observations. 


The exchange of heat between the Arctic Ocean and its sea ice is a complex process. I am proposing five ideas to improve our knowledge of the Arctic, its sea ice and its ocean:
\begin{enumerate}
\item High resolution simulations focussing on the junction between fluid and solid. This type of simulation would enhance our understanding of how heat and kinetic energy is transferred between the Arctic Ocean and the Arctic sea ice. 
\item A clear calculation reconstructing the CCSM temperature-heat equation. It seems the standard output does not offer the adequate variables to do so. The temperature is core in the many transformations the Arctic undergoes. It is mandatory to access the sources of the temperature increase if we want to understand the future of the Arctic. 
\item Climate model simulations with controlled ocean circulation and heat transport. The impact of the heat transport through the gates of the Arctic Ocean on the sea ice is a challenging topic. Only the first few ocean layers have the possibility to melt sea ice. It would be important to understand which layers are impactful and how deep they can penetrate into the Arctic Ocean. 
\item A simulated satellite measuring sea ice concentration in GCMs. In this era of Arctic sea ice melting, sea ice ponds are more and more present. This presence of water on sea ice is not detected by satellites who would register water instead of ice. A simulated satellite possessing the biases of real satellites would provide more comparable results to observations.
\item Climate simulations evolving through the future forcings instead of time. Several fields such as the sea ice volume behaves in a simpler way when plotted with the extra forcing instead of time. It could save computational time since one simulation through extra forcing forecasts all the future scenarios at once.
\end{enumerate}









%Even with an analytical solution of the climate equations, our future would be speculative. Due to the chaotic nature of the equations, a slight difference in the initial condition can lead to a completely different solution. The initial conditions are given by measurements. Those measurements are scarce and are not absolute, they have uncertainties. To obtain the solution for the climate of planet Earth, we need spatially continuous function of every variables at a given time which is impossible. So, even in the best case scenario we are left with partial answers. 

%Not having the analytical solution, we estimate it using global climate models. Those models solves the equations up to a certain scale defined by the model grid. Anything smaller than that the grid be parameterized. The models estimates the solution to the equations with incomplete initial conditions. What can we expect? What should we expect? How could we estimate the output of those models?

%GCMs require high computational power. With the exponential growth in computer power, more and more models, versions and simulations arise. I feel that the increase in model output is not match by the community ability to process them accordingly. Many papers treat of small subset of runs or consider only one run by model. The computer growth is not over and less valuable knowledge is gained from the gigantic amount of data we can access. I think it is time for our community to ponder on larger goals and revisit what we can and cannot do with the thousand simulations from hundreds of models that are available at this time. 




%Comparison of sea ice extent from model and from satellite is unfaithful. The sea ice extent calculated from model is exact; it does not have biases of error as the satellites with the melt ponds for example. Two options are possible to correctly compare both sea ice extent. The first one is to send and army of persons so monitor the sea ice concentration at the surface of near the surface and verify if seen water of part of the ocean of melt ponds. That way, both measures does not have major source of errors. The other option is to model the satellite inside the simulation and compare what the data from the modelled satellite with the real one. 













