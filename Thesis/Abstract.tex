\section*{Abstract}
\addcontentsline{toc}{section}{Abstract}

This thesis furthers the understanding thermal interactions between the ocean and the sea ice using the CCSM3. It focuses on the six simulations under the SRES A1B. \cite{ISI:000242942100008} argued that the rapid losses of September sea ice extent recorded in each of these simulations were preceded by pulses of warm water toward the Arctic Ocean one to two years beforehand. The presented analysis includes all the physical processes affecting sea ice in connection with the years of rapid September sea ice extent loss. Every melt processes increases until the years of rapid sea ice extent loss whereas the sea ice formation processes stagnate and the sea ice transport through the gates of the Arctic Ocean decreases. Therefore, the rapid loss of September sea ice extent is caused by increased melt. The basal melt is the process which increased the most. Averaged over the six simulations, the sea ice receives $20\,W/m^2$ in 1900  up to $120\,W/m^2$ from the ocean. Significantly more than the $2\,W/m^2$ predicted by \cite{ISI:A1971I688400007}. This important increase in heat transfer is caused by an exponential increase in sea surface temperature all over the Arctic Ocean. The heat sources causing the exponential increase in sea surface temperature for the top layer of the CCSM can be uncovered through a temperature-heat ocean budget. The temperature-heat ocean budget proved impossible to close satisfactorily even if using the extended output of the fourth version of the CCSM. The error over the grid columns exceeds $\pm 50\,W/m^2$. Nevertheless, the study of the heat transport through the gateways of the Arctic Ocean can provide insight on the warming of the Arctic Ocean and of its impact on sea ice melt. The heat transport through the gateways of the Arctic Ocean is studied for both the CCSM versions 3 and 4 by adding the results from five simulations under the RCP 6.0 scenario. Both models forecast very different Arctic conditions. The CCSM3 heat transport is dominated by the heat fluxes through the Barents Sea Opening. The total heat transfer to the Arctic Ocean is positive, warming it, from $35\,W/m^2$ in 1900 up to $130\,W/m^2$ in 2100. The CCSM4 heat transport has similar contributions from the different gateways to the Arctic Ocean. The total heat transfer simulated by the CCSM4 starts from $19\,W/m^2$ in 1900 up to $60\,W/m^2$ in 2100. The CCSM3 oceanic heat transfer is more than twice as important than the one simulated by the CCSM4.  


 






















